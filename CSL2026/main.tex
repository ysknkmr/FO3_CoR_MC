\documentclass[a4paper,UKenglish,cleveref, autoref, thm-restate]{lipics-v2021}

\usepackage{amsmath}
\usepackage{amsthm}
\usepackage{refcount} % ラベルの番号を取得するために使用
\newcounter{tempTheoremCounter} % カウンタを一時的に保存するための新しいカウンタを定義

\usepackage{tikz}
\usetikzlibrary{tikzmark, arrows.meta, bending}

\usepackage{CJKutf8} % 日本語を書きたいので
\usepackage{ascmac} % なんか枠を作る
\usepackage{stmaryrd}

\newcommand*{\Ja}[1]{%
  \begin{CJK}{UTF8}{ipxm}#1\end{CJK}%
}

\newenvironment{Jcomment}%
{\begin{screen}\begin{CJK}{UTF8}{ipxm}}%
{\end{CJK}\end{screen}}

\usepackage[most]{tcolorbox} % for \begin{tcolorbox}\end{tcolorbox}
% \usepackage[bibliography=common]{apxproof}
\usepackage{mathtools} % for \DeclarePairedDelimiter
\usepackage{environ} % for \NewEnviron

\usepackage[xcolor, cleveref, notion, quotation]{knowledge}
% \usepackage[xcolor, cleveref, notion, quotation, electronic]{knowledge}
\usepackage{mathcommand}
\knowledgeconfigure{protect quotation={figure,tikzpicture,tikzcd}} % [Y: added figure and tikzpicture for graphs library in TikZ]
\knowledgeconfigure{diagnose line=true, diagnose bar=true}

\definecolor{Dark Ruby Red}{HTML}{580507}
\definecolor{Dark Blue Sapphire}{HTML}{053641}
\definecolor{Dark Gamboge}{HTML}{be7c00}

\IfKnowledgePaperModeTF{
    %
}{
    % If we are NOT in paper mode (i.e. in composition mode or electronic mode)
    \knowledgestyle{intro notion}{color={Dark Ruby Red}, emphasize}
    \knowledgestyle{notion}{color={Dark Blue Sapphire}}
    % \hypersetup{
    %     colorlinks=true,
    %     breaklinks=true,
    %     linkcolor={cGreen}, % Links to sections, pages, etc.
    %     citecolor={cGreen}, % Links to bibliography
    %     filecolor={cGreen}, % Links to local file
    %     urlcolor={cGreen},
    % }
    \IfKnowledgeElectronicModeTF{
        %
    }{
        % If we are in composition mode, highlight unknown stuff (in yellow) and display the anchor point.
        % \knowledgeconfigure{anchor point color={Dark Ruby Red}, anchor point shape=corner}
        % \knowledgestyle{intro unknown}{color={Dark Gamboge}, emphasize}
        % \knowledgestyle{intro unknown cont}{color={Dark Gamboge}, emphasize}
        % \knowledgestyle{kl unknown}{color={Dark Gamboge}}
        % \knowledgestyle{kl unknown cont}{color={Dark Gamboge}}
    }
}

\usepackage[xcolor, cleveref, notion, quotation]{knowledge}
% \usepackage[xcolor, cleveref, notion, quotation, electronic]{knowledge}
\usepackage{mathcommand}
\knowledgeconfigure{protect quotation={figure,tikzpicture,tikzcd}} % [Y: added figure and tikzpicture for graphs library in TikZ]
\knowledgeconfigure{diagnose line=true, diagnose bar=true}

\definecolor{Dark Ruby Red}{HTML}{580507}
\definecolor{Dark Blue Sapphire}{HTML}{053641}
\definecolor{Dark Gamboge}{HTML}{be7c00}

\IfKnowledgePaperModeTF{
    %
}{
    % If we are NOT in paper mode (i.e. in composition mode or electronic mode)
    \knowledgestyle{intro notion}{color={Dark Ruby Red}, emphasize}
    \knowledgestyle{notion}{color={Dark Blue Sapphire}}
    % \hypersetup{
    %     colorlinks=true,
    %     breaklinks=true,
    %     linkcolor={cGreen}, % Links to sections, pages, etc.
    %     citecolor={cGreen}, % Links to bibliography
    %     filecolor={cGreen}, % Links to local file
    %     urlcolor={cGreen},
    % }
    \IfKnowledgeElectronicModeTF{
        %
    }{
        % If we are in composition mode, highlight unknown stuff (in yellow) and display the anchor point.
        % \knowledgeconfigure{anchor point color={Dark Ruby Red}, anchor point shape=corner}
        % \knowledgestyle{intro unknown}{color={Dark Gamboge}, emphasize}
        % \knowledgestyle{intro unknown cont}{color={Dark Gamboge}, emphasize}
        % \knowledgestyle{kl unknown}{color={Dark Gamboge}}
        % \knowledgestyle{kl unknown cont}{color={Dark Gamboge}}
    }
}


\newcommand{\fo}{\ensuremath{\mathbb{FO}}}
\DeclarePairedDelimiter{\sem}{\llbracket}{\rrbracket}
\DeclarePairedDelimiter{\set}{\{}{\}}
\DeclarePairedDelimiter{\tuple}{\langle}{\rangle}

\NewDocumentCommand\struc{O{1}}{%
    \ifcase#1
        undefined
    \or \mathfrak{A}
    \or \mathfrak{B}
    \else undefined

    \fi
}



\DeclarePairedDelimiter{\len}{\|}{\|}

\NewDocumentCommand\fml{O{1}}{%
    \ifcase#1
        undefined
    \or \varphi
    \or \psi
    \or \rho
    \else undefined

    \fi
}

%
% CoR
%
\knowledgenewrobustcmd{\union}{\mathbin{\cmdkl{+}}}
\knowledgenewrobustcmd{\intersection}{\mathbin{\cmdkl{\cap}}}
\knowledgenewrobustcmd{\compo}{\mathbin{\cmdkl{;}}}

\knowledge{text={i.e.\@}, italic}
  | ie
  | i.e.

\knowledge{text={I.e.\@}, italic}
  | Ie

\knowledge{text={s.t.\@}, italic}
  | st
  | s.t.

\knowledge{text={e.g.\@}, italic}
  | eg
  | e.g.

\knowledge{text={vs.\@}, italic}
  | vs

\knowledge{text={w.r.t.\@}, italic}
  | wrt
  | w.r.t

\knowledge{text={a.k.a.\@}, italic}
  | aka
  | a.k.a

\knowledge{text={w.l.o.g.\@}, italic}
  | wlog
  | w.l.o.g.

\knowledge{text={W.l.o.g.\@}, italic}
  | Wlog
  | W.l.o.g.

\knowledge{text={cf.\@}, italic}
  | cf
  | cf.

\knowledge{text={iff}, italic}
  | iff

\knowledge{text={r.e.\@}, italic}
  | r.e.
  | re

\knowledge{text={\ensuremath{\text{\textsc{ExpTime}}}}}
  | EXPTIME
  | ExpTime

\knowledge{text={\ensuremath{\text{\textsc{PSpace}}}}}
  | PSPACE
  | ExpTime

\knowledge{notion}
| SETH
| Strong Exponential Time Hypothesis

\knowledge{notion}
| Williams' algorithm
%| Williams--Gao--Impagliazzo's algorithm

%
% fo
%
\knowledge{notion}
| model checking problem
| combined complexity model checking problem

\knowledge{notion}
| signature
| signatures
| relational signature
| relational signatures

\knowledge{notion}
| structure
| structures

\knowledge{notion}
| universe

\knowledge{notion}
| vertices
| vertex

\knowledge{notion}
| variable
| variables
| variable@fo
| variables@fo

\knowledge{notion}
| formula
| formulas

\knowledge{notion}
| sentence
| sentences

\knowledge{notion}
| prenex normal form
| PNF

%
% CoR
% 
\knowledge{notion}
| Tarski's calculus of relations
| CoR

\knowledge{notion}
| composition
| relational composition

%
% propositional
%
\knowledge{notion}
| CNF
| conjunctive normal form

\knowledge{notion}
| DNF
| disjunctive normal form

\knowledge{notion}
| SAT

\knowledge{notion}
 | formula@propositional
 | formulas@propositional

\knowledge{notion}
 | variable@propositional
 | variables@propositional

\knowledge{notion}
 | clause
 | clauses
 | clause@propositional
 | clauses@propositional

\knowledge{notion}
 | length@propositional


\bibliographystyle{plainurl}% the mandatory bibstyle

\title{Tightening Faster Model Checking of First-Order Graph Properties, via Tarski's Calculus of Relations \begin{sideyoshiki}Tentative\end{sideyoshiki}} %TODO Please add
% On Faster Model Checking of First-Order Graph Properties and Tarski's Calculus of Relations

\titlerunning{FO3}

\author{Yoshiki Nakamura}{Science Tokyo, Japan}{mail}{https://orcid.org/0000-0003-4106-0408}{}
\author{Yuya Uezato}{CyberAgent, Inc., Japan}{mail}{https://orcid.org/0009-0005-8834-010X}{}

\authorrunning{Y. Nakamura and Y. Uezato}

\Copyright{Yoshiki Nakamura and Yuya Uezato}

\ccsdesc[500]{Theory of computation~Logic}

\keywords{First-order Logic,Fine-grained Complexity}

\category{} %optional, e.g. invited paper

\relatedversion{}

%\nolinenumbers %uncomment to disable line numbering

\begin{document}


\maketitle

\begin{abstract}
This is a SUPER-STRONG-PAPER and brings a NEW ERA to the field of FO.

\end{abstract}

\section{Notation}

\begin{hidden}
\Ja{
(メモ:
LIPICSとpLaTeXは異常に組み合わせが悪いので、
コメントなどで日本語を書く時はこのようにします。)}

\begin{screen}
Cute round-corner box \\
\Ja{コメントを付ける時には、screenで囲うとこういう感じになります}
\end{screen}

\begin{yoshiki}
  \Ja{コメントテスト}
\end{yoshiki}

\begin{sideyoshiki}
  \Ja{コメントテスト2}
\end{sideyoshiki}
\end{hidden}

We write $\fo^k$ to denote the set of formulas of First-order predicate logic with $k$-variables.

\paragraph*{Important Remark (Unary and Binary Predicates)}
We allow unary and binary predicates in our $\fo^3$.
Therefore, we do not allow terms involving ternary predicates such as $\forall x.\,\exists y.\,\forall z.\, P(x, y, z)$.%
\footnote{For each unary predicate $U$, we can simulate it by a diagonal binary predicate $B_U(x, y) := x = y \land U(x)$. However, in our later construction, we use some unary predicates; thus, we also allow unary predicates explicitly.}

On this remark, we can identify each model $\mathcal{M}$ as color graphs as follows:
\begin{itemize}
\item Each element corresponds a node (vertex).
\item Each unary predicate $U_i$ means ``color-$i$'' node.
\item Each binary predicate $P_j$ is ``color-$j$'' edge.
\item For example, on two nodes $x, y$, a term $(U_i(x) \land U_j(y) \land P_k(x, y))$ means that $x$ has the color $i$, $y$ has the color $j$, and there is a color $k$ edge from $x$ to $y$.
\end{itemize}

\paragraph*{Quantifier width}
\begin{Jcomment}
Quantifier widthという謎の指標を導入します。

与えられた式の部分項 $\forall/\exists x. ( \cdots )$ について $\cdots$部分に「直接」出てくるquantifierの個数を、この部分項のquantifier widthと呼ぶことにします。

たとえば、次の部分項
$$
\forall x.\biggl( \Bigl(\underline{{\color{red}\forall}} y. P(y)\Bigr) \lor \Bigl(\underline{{\color{red}\forall}} z. (\underline{\forall} x. P(z, x)) \land Q(z)\Bigr) \biggr)
$$
については、内部に3つのquantifierの出現(下線)があるのですが、直接の出現は赤くした2つだけなので、この部分項のquantifier widthは「2」です。

そして、式のquantifier widthというのは、最大のquantifier widthを持つ部分項における、その数として定義します。
\end{Jcomment}

\section{Introduction}\label{sec: introduction}
\begin{yoshiki}
    TODO: (Below is wip.)
\end{yoshiki}

The ""model checking problem"" is the following problem:
\begin{center}
given both a "sentence" $\fml$ and a "structure" $\struc$, does $\struc \models \fml$ hold?
\end{center}
Unfortunately, the "model checking problem" for $\fo$ is complete in "PSPACE".
Nevertheless, when the number of "variables" is fixed, the "model checking problem" for $\fo^{k}$ can be decided in $\mathcal{O}(\len{\fml} \cdot n^{k})$ time by a naive bottom-up evaluation algorithm (see, "eg", \cite[Proposition 3.1]{vardiComplexityBoundedvariableQueries1995}),
where $\len{\fml}$ is the length of $\fml$ and $n$ is the number of \kl{vertices} in $\struc$.
A natural question is 
\begin{center}
    Can the complexity be improved from $\mathcal{O}(\len{\fml} \cdot n^{k})$ time?
\end{center}

\begin{yoshiki}
    TODO: Parameterized complexity ("eg", \cite{flumParameterizedComplexityTheory2006,flumFixedParameterTractabilityDefinability2001,flumModelCheckingProblemsBasis2005, frickComplexityFirstorderMonadic2004}).

    Parameterized complexity for space complexity \cite{chenParameterizedSpaceComplexity2019}
\end{yoshiki}

Williams \cite{williamsFasterDecisionFirstorder2014} gave a positive answer to this question.
He showed that when $\fml$ is \emph{fixed},
\begin{itemize}
    \item For $\fo^{3}$ \kl{sentences}\footnote{Precisely, Williams \cite{williamsFasterDecisionFirstorder2014} shows that for $\fo^{k}$ \kl{sentences} in the \emph{"prenex normal form"} (so, the number of occurrences of "quantifiers" is at most $k$),
    the "model checking problem" is in $\mathcal{O}(n^{k - 3 + \omega})$ time for $k \ge 3$ \cite[Corollary 1.3]{williamsFasterDecisionFirstorder2014},
    and in $n^{k-1 + o(1)}$ time for $k \ge 9$ \cite[Theorem 1.3]{williamsFasterDecisionFirstorder2014}.
    Gao and Impagliazzo \cite[Theorem 7]{gaoFineGrainedComplexityStrengthenings2019} claims that
    the algorithm can be extended for general $\fo^{k}$ \kl{sentences} when $k = 3$.},
    the "model checking problem" can be decided in $\mathcal{O}(n^{\omega})$ time \cite[Theorem 1.2]{williamsFasterDecisionFirstorder2014}\cite[Theorem 7]{gaoFineGrainedComplexityStrengthenings2019},
    where $\omega$ is the exponent of matrix multiplication.%
    
    \item For $\fo^{k}$ \kl{sentences}\footnote{This claims holds even for $\fo^{k}$ \kl{sentences} in $\exists^{*} \forall$-"prenex normal form".} where $k \ge 4$,
    if the "model checking problem" can be decided in $\mathcal{O}(n^{k-1-\varepsilon})$ time for some $\varepsilon > 0$,
    then the "Strong Exponential Time Hypothesis" ("SETH") is false \cite[Corollary 1.4]{williamsFasterDecisionFirstorder2014}.
\end{itemize}

However when $\fml$ is not fixed,
the "Williams--Gao--Impagliazzo's algorithm" \cite{williamsFasterDecisionFirstorder2014,gaoFineGrainedComplexityStrengthenings2019} for $\fo^{3}$ \kl{sentences} is $\mathcal{O}(2^{\len{\fml}} \cdot n^{\omega})$ time.
The exponential blowup "wrt" $\len{\fml}$ is due to that the algorithm requires a transformation to the "disjunctive normal form" ("DNF").

...

In this paper, we first revisit the "Williams--Gao--Impagliazzo's algorithm".
We give a perspective from ""Tarski's calculus of relations"" (henceforth, \reintro*\kl{CoR}) \cite{tarskiCalculusRelations1941}.
\dots
We can translate $\fo^{3}$ \kl{sentences} into \kl{CoR} \kl{sentences} in $\mathcal{O}(2^{\len{\fml}})$ time.
The original translation is given in \cite{tarskiFormalizationSetTheory1987}.
An $\mathcal{O}(2^{\len{\fml}})$ time translation is given in \cite{nakamuraExpressivePowerSuccinctness2022}; an implementation of the translation \cite{nakamuraExpressivePowerSuccinctness2020,nakamuraExpressivePowerSuccinctness2022} can be found in \cite{brogniTranslatingFirstOrderPredicate2023}.\footnote{"Williams--Gao--Impagliazzo's algorithm" takes almost the same steps as in the translation in \cite{nakamuraExpressivePowerSuccinctness2022}... (TODO: revise)}
For \kl{CoR} \kl{sentences}, the "model checking problem" can be decided in $\mathcal{O}(\len{\fml} \cdot n^{\omega})$ time by a naive bottom-up evaluation algorithm,
where we use the "matrix multiplication" algorithm for the "relational composition" $\compo$.
Combining them, we have that the "model checking problem" for $\fo^{3}$ \kl{sentences} can be decided in $\mathcal{O}(2^{\len{\fml}} \cdot n^{\omega})$ time.


Now, 
...


\section{Preliminaries}\label{sec: preliminaries}

% \section{Notation}

% \begin{hidden}
% \Ja{
% (メモ:
% LIPICSとpLaTeXは異常に組み合わせが悪いので、
% コメントなどで日本語を書く時はこのようにします。)}

% \begin{screen}
% Cute round-corner box \\
% \Ja{コメントを付ける時には、screenで囲うとこういう感じになります}
% \end{screen}

% \begin{yoshiki}
%   \Ja{コメントテスト}
% \end{yoshiki}

% \begin{sideyoshiki}
%   \Ja{コメントテスト2}
% \end{sideyoshiki}
% \end{hidden}

\AP We write $\Z$ for the set of all integers.

\AP We write $\range{l, r}$ for the set $\set{i \in \Z \mid l \le i \le r}$.

\AP A ""relational signature"" (henceforth, "signature") $\sig$ is ...

\AP A ""structure"" $\struc$ over a "signature" $\sig$ is a tuple $\tuple{\univ{\struc}, \set{R^{\struc}}_{R \in \sig}}$,
where its ""universe"" $\univ{\struc}$ is a finite\footnote{In this paper, we are only interested in finite "structures".} set of ""vertices""
and each $R^{\struc}$ is a binary relation on $\univ{\struc}$.

\AP A ""sentence"" is a "formula" without free "variables@@propositional".

For a "formula" $\fml$ and a "valuation" $\mathfrak{v}$, we write 
$\struc, \mathfrak{v} \models \fml$ to denote that $\fml$ is true at $\struc$ under $\mathfrak{v}$.
For a "sentence" $\fml$, we write $\struc \models \fml$ to denote that $\fml$ is true at $\struc$.

We write $\fo^k$ to denote the set of formulas of First-order predicate logic with $k$-variables.

\paragraph*{Important Remark (Unary and Binary Predicates)}
We allow unary and binary predicates in our $\fo^3$.
Therefore, we do not allow terms involving ternary predicates such as $\forall x.\,\exists y.\,\forall z.\, P(x, y, z)$.%
\footnote{For each unary predicate $U$, we can simulate it by a diagonal binary predicate $B_U(x, y) := x = y \land U(x)$. However, in our later construction, we use some unary predicates; thus, we also allow unary predicates explicitly.}

On this remark, we can identify each model $\mathcal{M}$ as color graphs as follows:
\begin{itemize}
\item Each element corresponds a node (vertex).
\item Each unary predicate $U_i$ means ``color-$i$'' node.
\item Each binary predicate $P_j$ is ``color-$j$'' edge.
\item For example, on two nodes $x, y$, a term $(U_i(x) \land U_j(y) \land P_k(x, y))$ means that $x$ has the color $i$, $y$ has the color $j$, and there is a color $k$ edge from $x$ to $y$.
\end{itemize}

\paragraph*{Quantifier width}
\begin{Jcomment}
Quantifier widthという謎の指標を導入します。

与えられた式の部分項 $\forall/\exists x. ( \cdots )$ について $\cdots$部分に「直接」出てくるquantifierの個数を、この部分項のquantifier widthと呼ぶことにします。

たとえば、次の部分項
$$
\forall x.\biggl( \Bigl(\underline{{\color{red}\forall}} y. P(y)\Bigr) \lor \Bigl(\underline{{\color{red}\forall}} z. (\underline{\forall} x. P(z, x)) \land Q(z)\Bigr) \biggr)
$$
については、内部に3つのquantifierの出現(下線)があるのですが、直接の出現は赤くした2つだけなので、この部分項のquantifier widthは「2」です。

そして、式のquantifier widthというのは、最大のquantifier widthを持つ部分項における、その数として定義します。
\end{Jcomment}

\paragraph*{Tarski's Calculus of Relations (CoR)}
"CoR" "terms" are generated by the following grammar:
\begin{align*}
    \term[1], \term[2] \;\Coloneqq\;
    & a \mid \term[1] \union \term[2] \mid \term[1] \intersection \term[2] \mid \term[1]^{-}\\
    & \phantom{a} \mid \term[1] \compo \term[2] \mid \term[1] \dagger \term[2] \mid \term[1]^{\pi}.  
\end{align*}

\begin{proposition}\label{proposition: CoR model checking}
    The "model checking problem" for "atomic" "CoR" "formulas" is in $\mathcal{O}(\len{\fml} \cdot n^{\omega})$ time.
\end{proposition}
\begin{proof}[Proof Sketch]
    By a naive bottom-up evaluation algorithm.
    Here, we use the "matrix multiplication" algorithm for the "relational composition" ($\compo$).
\end{proof}

\paragraph*{SETH}

\begin{conjecture}[The ""Strong Exponential Time Hypothesis"" (""SETH"") \cite{impagliazzoComplexity$k$SAT2001,impagliazzoWhichProblemsHave2001}]\label{conjecture: SETH}
    For every $\delta < 1$,
    there exists an integer $k$ such that
    "$k$-CNF"-"SAT" with $n$ "variables@@propositional" cannot be solved in $\mathcal{O}(2^{\delta n})$ time.%
    \begin{sideyoshiki}
    \Ja{たとえば \cite{vassilevskawilliamsHardnessEasyProblems2015} では、$\mathcal{O}(2^{\delta n})$ でなく $2^{\delta n} \poly (n)$ を用いているが、}\\
    \Ja{予想としては同値(のはず。$\poly(n) < \mathcal{O}(2^{\delta n})$ なので)。}\\
    \Ja{$\mathcal{O}(2^{n} \poly(n))$ のアルゴリズムが存在するという fact に合わせている?} 
    \end{sideyoshiki}
\end{conjecture}
""SETH"" implies the following hypothesis.
\begin{sideyoshiki}\label{conjecture: CNF-SAT}
    \Ja{However, it is not known whether the two conjectures are equivalent.}
    \cite[Previous Work]{cyganProblemsHardCNFSAT2016}
\end{sideyoshiki}
\begin{conjecture}["eg", in {\cite{patrascuPossibilityFasterSAT2010}}]
    For every $\delta < 1$,
    "CNF"-"SAT" with $n$ "variables@@propositional" and $m$ "clauses" cannot be solved in $2^{\delta n} \cdot \poly(m)$ time.
\end{conjecture}



\section{FPT on FO3}

We write $\fo^3_{k, w}$ to denote the set of formulas of $\fo^3$ with $k$-predicates and $w$-quantifier-width.

%\begin{restatable}[Number of predicates is a FPT-parameter]{theorem}{thmFOthreeFPT}
\begin{theorem}[\text{$\fo^3_{k, w}$ model checkig is FPT}]
\label{thm:FO3-FPT}
\mbox{}

Let $\varphi : \fo^3_{k, w}$.

For a given model (graph) $\mathcal{G}$,
we can decide if $\mathcal{G} \models \varphi$ in $O(2^{k + w} \cdot |\varphi| \cdot N^{\omega})$ where
\begin{itemize}
\item $N$ is the number of vertices in $\mathcal{G}$; and
\item $\omega$ is a complexity constant for the boolean matrix multiplication (BMM): i.e., $O(N^\omega)$-time is the time complexity for BMM of two $(N, N)$ matrices.\footnote{\Ja{何らかの論文をciteして、今の記録がどれくらいか書いておくと分かりやすい}}
\end{itemize}
\end{theorem}
 %\end{restatable}

\begin{corollary}
If we fix the number of predicates $k$ and width $w$,
the model checking problem $\mathcal{G} \models \varphi$ in $O(|\varphi| \cdot N^{\omega})$ time-complexity where $N$ is the number of nodes in $\mathcal{G}$.
\end{corollary}

\begin{Jcomment}
    $k$をfixしない、素朴な $\set{ (x, y, z) }$ 構築アルゴリズムだと $O(|\varphi| \cdot N^3)$ になって、
    その設定だと Theorem~\ref{thm:FO3-FPT} は 係数 が大きくなりすぎて遅いということを、
    意味のある形で言及した方が良い。
\end{Jcomment}

\Ja{以降、Theorem~\ref{thm:FO3-FPT} の証明をやっていきます。}

\subsection{Proof of Theorem~\ref{thm:FO3-FPT}}

Here we use $P, Q, R \ldots$ to denote binary predicates and $U, \ldots$ for unary predicates.

For a given graph (model) $\mathcal{G}$, we write $N_{\mathcal{G}}$ or simply $N$ to denote the number of vertices of $\mathcal{G}$.

\paragraph*{Our idea: Quantifier elimination (Simple case)}

Our query evaluation strategy is evaluating the innermost (quantifier-free formula) to the outermost along with quantifier eliminating.

Let us consider the following situation:
$$
Qx. \cdots
( Qy. \cdots
  ( \tikzmarknode{Qz}{Qz}. \cdots
    ( Qx. \cdots
      ( \tikzmarknode{Qy}{Qy}. \cdots
        ( \exists\tikzmarknode{Ex}{x} .
          ( 
            P(\tikzmarknode{Ry}{y}, \tikzmarknode{Rx}{x})
            \land          
            R(\tikzmarknode{Px}{x}, \tikzmarknode{Pz}{z})
          )
        )
      )
    )
  )
)
$$
%
% 矢印の描画
\begin{tikzpicture}[
    remember picture,
    overlay,
    arr/.style={-{[bend]Stealth[length=2mm, width=1.5mm]}, shorten <=0.5pt, shorten >=0.5pt}
]
    % P(x,z) の x から ∃x へ (赤色の矢印)
    \draw[arr, red]
        (Px.north) .. controls +(0, 0.4) and +(0, 0.4) .. (Ex.north);

    % R(y,x) の x から ∃x へ (赤色の矢印)
    \draw[arr, red]
        (Rx.north) .. controls +(0, 0.6) and +(0, 0.6) .. (Ex.north);

    % P(x,z) の z から Qz へ (青色の矢印)
    \draw[arr, blue]
        (Pz.north) .. controls +(0, 0.8) and +(0, 0.8) .. (Qz.north);
        
    % R(y,x) の y から Qy へ (緑色の矢印)
    \draw[arr, green!60!black]
        (Ry.north) .. controls +(0, 0.6) and +(0, 0.6) .. (Qy.north);
\end{tikzpicture}
To eliminate $\exists z. P(y, x) \land R(x, z)$, we introduce a new predicate defined as follows:
$$
S(x, y) := \exists z. P(x, z) \land Q(z, y).
$$

It is worth noting that $S$ is just the composed predicate of $P$ and $Q$.
So, we write $P \odot Q$ instead of $S$.

Using this predicate, we can rewrite the above one to the following one:
$$
Qx \cdots
( Qy \cdots
  ( \tikzmarknode{Qz}{Qz} \cdots
    ( Qx.\
      (
        \tikzmarknode{Qy}{Qy}.\ (P \odot R)(\tikzmarknode{ty}{y}, \tikzmarknode{tz}{z})
      )
    )
  )
)
$$

\begin{tikzpicture}[
    remember picture,
    overlay,
    arr/.style={-{[bend]Stealth[length=2mm, width=1.5mm]}, shorten <=0.5pt, shorten >=0.5pt}
]
    % P(x,z) の z から Qz へ (青色の矢印)
    \draw[arr, blue]
        (tz.north) .. controls +(0, 0.6) and +(0, 0.6) .. (Qz.north);
        
    % R(y,x) の y から Qy へ (緑色の矢印)
    \draw[arr, green!60!black]
        (ty.north) .. controls +(0, 0.4) and +(0, 0.4) .. (Qy.north);
\end{tikzpicture}

Continuing this process, if we eventually enter $\forall x. \exists y. T(x, y)$:
\begin{enumerate}
\item building a unary predicate $T_y(x) := \exists y. T(x, y)$ in $O(N^2)$-time, we rewrite it to $\forall x. T_y(x)$.
\item we then simply evaluate $\forall x. T_y(x)$ as $\bigwedge_{1 \leq i \leq N} T_y(i)$ in $O(N)$-time.
\end{enumerate}

\paragraph*{Time-complexity of basic operations}

Here we enumerate the time-complexity of our basic operations, which will be used later:
\begin{itemize}
\item Building a new predicate $Q(x, y) := P(x, y) \bowtie R(x, y)$ ($\mathord{\bowtie} \in \set{ \land, \lor }$) needs $O(N^2)$-time.
%
\item Transposing a predicate $P^T(x, y) := P(y, x)$ needs $O(N^2)$-time.
%
\item Evaluating unary predicates $\mathcal{Q} x. P(x)$ ($\mathcal{Q} \in \set{\forall, \exists}$) needs $O(N)$-time.
%
\item Building a new predicate $Q(x) := \mathcal{Q} y. P(x, y)$ ($\mathcal{Q} \in \set{\exists, \forall}$) needs $O(N^2)$-time.
\end{itemize}

So, the remaining operator is predicate composing $(P \odot Q)(x, y) := \exists z. P(x, z) \land Q(z, y)$ from predicates $P$ and $Q$.
Although a very simple matrix multiplication requires $O(N^3)$-time,
we can build such $P \odot Q$ using fast matrix multiplication algorithms.
\begin{proposition}[Fact]
From two predicates $P$ and $Q$, we can build the composited predicate $P \odot Q$ in $O(N^\omega)$-time.
\end{proposition}

\subsection{Quantifier elimination: General case}

If we encounter terms of the very restricted form like $\exists z. P(x, z) \land Q(z, y)$,
we can replace it by $(P \odot Q)(x, y)$ with eliminating the quantifier occurrence $\exists z$.

In other words, if we encounter more general form of $\exists z. \cdots$, we need to translate it to some adequate form for quantifier elimination.
%
Let us consider the following term:
$$
\exists z. \biggl(
\bigl(P(x, z) \lor P(y, z) \lor P(x, y) \bigr)
\land
\bigl(Q(x, z) \lor Q(x, y)\bigr)
\land
\bigl(R(y, z) \lor R(x, y)\bigr)
\biggr).
$$

We cannot directly apply our quantifier elimination strategy because $\exists$-quantifiers do not distribute on $\land$: i.e.,
$(\exists z. \varphi_1 \land \varphi_2) \neq (\exists z. \varphi_1) \land (\exists z. \varphi_2)$.

To apply our quantifier elimination procedure, DNF is an adequate form.
However, as well-known, converting-to-DNF translation generates an exponential-large expression in the worst case.
For our example, we fist translate the first $(\cdots) \land (\cdots)$ subterm as follows:
$$
\begin{array}{l}
\bigl(P(x, z) \lor P(y, z) \lor P(x, y) \bigr)
\land
\bigl(Q(x, z) \lor Q(x, y)\bigr) \\
\quad \Rightarrow \\
\phantom{\lor}\,
(P(x, z) \land Q(x,z)) \lor
(P(y, z) \land Q(x,z)) \lor
(P(x, y) \land Q(x, z)) \\
\lor\,
(P(x, z) \land Q(x, y)) \lor
(P(y, z) \land Q(x, y)) \lor
(P(x, y) \land Q(x, y)).
\end{array}
$$
We continue to normalize this term with $R(y, z) \lor R(x, y)$
and it generates DNF with 12 bases of the form $P(\_, \_) \land Q(\_, \_)$.

Since our goal is to design an $O(F(k, w) \cdot |\varphi| \cdot N^{\omega})$-time algorithm, we cannot adopt converting-to-DNF transformations.


\subsection{Tseytin-like (?) transformation for General Cases}

As we have seen above, explicit translations to DNF does not work well since it may explode given expression to an exponential size.

We here develop a translation inspired by Tseytin's transformation.\footnote{\url{https://en.wikipedia.org/wiki/Tseytin_transformation}}

\begin{Jcomment}
    実際はインスパイアされたという程でもないんですが、
    とはいえよく知られた構成な気もするので、何かciteできるものがあればしたいという感じ。
\end{Jcomment}

Let us revisit the above example:
$$
\exists z. \biggl(
\bigl(P(x, z) \lor P(y, z) \lor P(x, y) \bigr)
\land
\bigl(Q(x, z) \lor Q(x, y)\bigr)
\land
\bigl(R(y, z) \lor R(x, y)\bigr)
\biggr).
$$

First, we generate all possible \emph{assignments} to predicates.
For our example, we have the following assignments:
$$
\begin{array}{c|ccccccc}
& P(x, z) & P(y, z) & P(x, y) & Q(x, z) & Q(x, y) & R(y, z) & R(x, y) \\\hline
\alpha_1 & 0 & 0 & 0 & 0 & 0 & 0 & 0 \\
\alpha_2 & 0 & 0 & 0 & 0 & 0 & 0 & 1 \\
\alpha_3 & 0 & 0 & 0 & 0 & 0 & 1 & 0 \\
\vdots \\
\alpha   & 1 & 0 & 1 & 0 & 1 & 0 & 0 \\
\vdots \\
\alpha_{\star}   & 1 & 1 & 1 & 1 & 1 & 1 & 1
\end{array}
$$

Let $\mathcal{A}$ be the set of all assignments.

Let $\sem{\Psi}_{\alpha}$ be a boolean value obtained by evaluationg $\Psi$ under an assignment $\alpha$.

For our example expression $\psi$, $\sem{\psi}_{\alpha} = 0$ using $\alpha$ appearing in the above table.

If we change $\alpha$ as $\alpha(R(y,z)) \gets 1$, then $\sem{\psi}_{\alpha} = 1$.

Introducing assignment leads to the following useful proposition.
\begin{proposition}
$$
\Psi(x, y, z) \iff
\bigvee_{\alpha \in \mathcal{A}} \Bigl( \alpha \land \sem{\Psi}_{\alpha} \Bigr).
$$

Especially, let $\mathcal{V} = \set{ \alpha \in \mathcal{A} : \sem{\Psi}_\alpha }$, then we have:
$$
\Psi(x,y, z) \iff \bigvee_{\alpha \in \mathcal{V}} \alpha(x, y, z). \qquad (\star)
$$
\end{proposition}


By the definition of assignments, $\alpha$ takes the following form:
$$
\alpha(x, y, z) \equiv P_1(x, y) \land P_1(x, z) \land P_2(y, z) \land \cdots \land P_k(x, z).
$$
It means that the right-hand side of $(\star)$ takes DNF. Therefore,
$$
\exists z. \Psi(x, y, z) \iff (\exists z. \alpha_1(x, y, z)) \lor (\exists z. \alpha_2(x, y, z)) \lor \cdots \lor (\exists z. \alpha_{|\mathcal{V}|}(x, y, z)).
$$
It suffices to eliminating an $\exists$-quantifier from $\exists z. \alpha(x, y, z)$ for an assignment $\alpha$.

We first divide $\alpha(x, y, z)$ into two parts $\alpha_z$ and $\alpha_{-z}$ as follows:
$$
\alpha(x, y, z) \equiv \underbrace{(P_1(x, z) \land P_2(z, x) \land P_1(y, z) \land \cdots)}_{\alpha_z : z-\text{involving-terms}} \land \underbrace{(P_1(x, y) \land P_1(y, x) \land P_2 \land \cdots)}_{\alpha_{-z} : z-\text{non-involving-terms}}
$$
We now have the following:
$$
\Bigl(\exists z. \alpha(x, y, z)\Bigr) \iff \Bigl(\alpha_{-z}(x, y) \land (\exists z. \alpha_z(x, y, z))\Bigr).
$$

Finally, we transform $\alpha_z$ in the following steps (order):
\begin{enumerate}
\item If $\alpha_z$ has a term $P(z, x)$, we replace it with $\widetilde{P}(x, z)$ where $\widetilde{P}(a, b) = P(b, a)$.
\item Similarly, we change terms of the form $P(y, z)$ to $\widetilde{P}(z, y)$.
\item At this point, 
$$
\alpha_z(x, y, z) \equiv \underbrace{(P_1(x, z) \land P_3(x, z) \land P_4(x, z) \land \cdots)}_{\Psi_x} \land \underbrace{(P_2(z, y) \land P_3(z, y) \land P_5(z, y) \land \cdots)}_{\Psi_y}.
$$
\item Let $\Psi_x$ (resp. $\Psi_y$) be the set of predicates appearing in the above former (resp. latter) part.
\item Let us introduce two new predicates:
$$
\pi(x, z) := \bigwedge_{P \in \Psi_x} P(x, z), \qquad
\lambda(z, y) := \bigwedge_{Q \in \Psi_y} Q(z, y).
$$
\item We reach our goal:
$$
\alpha_z(x, y, z) = \pi(x, z) \land \lambda(z, y).
$$
\end{enumerate}

Using the final form, we obtain the following:
$$
\begin{array}{lcl}
\biggl(\exists z. \alpha(x, y, z)\biggr) & \iff &
\biggl(\alpha_{-z}(x, y) \land (\exists z. \pi(x, z) \land \lambda(z, y))\biggr) \\
& \iff & \biggl(\alpha_{-z}(x, y) \land (\pi \odot \lambda)(x, y)\biggr) \\
& \iff & (\alpha_{-z} \land (\pi \odot \lambda))(x, y)
\end{array}
$$

\subsection{Considering Complexity}
To eliminate an $\exists$-quantifier from an assignment $\alpha$, we introduce a new single predicate $\alpha_{-z} \land (\pi \odot \lambda)$.

Now we have a question:
To eliminate the $\exists$-quantifier from $\exists z. \Psi(x, y, z)$,
from the following characterization
$$
\biggl(\exists z. \Psi(x,y, z)\biggr)
\iff \biggl(\exists z. \bigvee_{\alpha \in \mathcal{V}} \alpha(x, y, z)\biggr)
\iff \bigvee_{\alpha \in \mathcal{V}} \biggl(\exists z. \alpha(x, y, z)\biggr)
\iff \bigvee_{\alpha \in \mathcal{V}} (\alpha_{-z} \land (\pi \odot \lambda))(x, y)
$$
can we avoid to introduce $O(2^k)$ predicates if we have $k$-predicates??
Please recall that each assignment is a choice from $9k$-literals from $\mathcal{L}$ where
$$
\mathcal{L} = \set{ Q(a, b) : a, b \in \set{x, y, z}, Q \in \set{P_1, P_2, \ldots, P_k} }.
$$

\textbf{Yes. We can avoid introducing a large number of predicates} because
we just need to introduce the following single (but large) predicate:
$$
(\alpha^1_{-z} \land (\pi^1 \odot \lambda^1)) \lor (\alpha^2_{-z} \land (\pi^2 \odot \lambda^2)) \lor \cdots \lor (\alpha^n_{-z} \land (\pi^n \odot \lambda^n))
$$
for $\mathcal{V} = \set{ \alpha^1, \alpha^2, \ldots, \alpha^n }$.

Let us consider what happens when repeatedly eliminating quantifiers from the innermost to the outermost:
$$
\begin{array}{l}
\exists y.
\Bigl(
(\exists z. \cdots) \bowtie (\exists z. \cdots) \bowtie \cdots \bowtie (\exists z. \cdots)
\Bigr) \\
\quad \Rightarrow \\
\exists y. \Bigl(
\begin{array}{c}
\text{Q. how many predicates appear in this scope?} \\
\text{A. $k + w$ where $w$ is the quantifier width of $\psi$}
\end{array}
\Bigr) \\
\quad \Rightarrow \\
\text{Eliminating $(\exists y.)$ introduces a single new predicate.}
\end{array}
$$

Since the required number of quantifier elimination (= the quantifier depth) is bounded by the size of a given expression $|\psi|$, we obtain the following our theorem in the end.

% \thmFOthreeFPT*

% 1. 現在の定理カウンタの値を保存
\setcounter{tempTheoremCounter}{\value{theorem}}
% 2. 目的の定理番号より1つ前の値にカウンタを設定
\setcounter{theorem}{\numexpr\getrefnumber{thm:FO3-FPT}-1\relax}
% 3. 再掲出
\begin{theorem}[Restatement]
Let $\varphi : \fo^3_{k, w}$.

For a given model (graph) $\mathcal{G}$,
we can decide if $\mathcal{G} \models \varphi$ in $O(2^{k+w} \cdot |\varphi| \cdot N^{\omega})$.
\end{theorem}
% 4. カウンタを元の値に戻す
\setcounter{theorem}{\value{tempTheoremCounter}}

\section{FO2 model checking}

\begin{Jcomment}
    $\fo^2$の論理式 $\psi$ と、グラフ $\mathcal{G}$ について
    $\mathcal{G} \models \psi$ のモデル検査問題は
    $O(|\psi| \cdot |\mathcal{G}|)$ で解ける気がします。
    ただし、$|\mathcal{G}|$は頂点数ではなく、述語(つまり辺)の定義の記述長も合わせたものです。
    面白そうなら書いても良いかと思うのですが、$\fo^3$の話だけにしたほうが、話がぼやけないならなくても良いかなという気がします。

    Dense graphについては $(x, y)$ の値の集合を構築する方法で $O(|\psi| \cdot N^2)$ になります。
    この $N$ は $\mathcal{G}$ の頂点数の方です。
    ただ、sparse graphの時には $N^2$ は$|\mathcal{G}|$について線形ではないので、そこを改良したいという感じです。

    アイディアは、$\fo^3$の時と比べると、$\exists y. P(x, y) \land Q(y, x) \simeq (P \odot Q)$ の計算について $(P \odot Q)(x, x)$ の diagonal な形しか要求しないので、行列積を全部やる必要が多分なさそうとかそういう感じです。

    Sparse graphの話なので、もしかすると Impagliazzo たちがやっているかもしれません。
\end{Jcomment}

\begin{yoshiki}
\Ja{\cite{gaoCompletenessFirstorderProperties2018} Lemma 9.1 Base Case は $O(|\psi| \cdot |\mathcal{G}|)$ ??}
\end{yoshiki}
\section{from }


\cite{williamsFasterDecisionFirstorder2014}
\section{Complexity gap between FO3 and CoR}\label{sec: from cnf}
% Is the exponential grow up of \Cref{theorem: fo3 model checking} unavoidable?
A natural question arising from \Cref{theorem: fo3 model checking} is whether there is a faster algorithm without the exponential blowup with respect to the length of the input sentence.
Namely,
\begin{center}
    Can we give an $\poly(\len{\fml}) \cdot n^{3 - \eps}$ time algorithm for the "model checking problem" for $\fo^{3}$?
\end{center}
In this section, we answer to this question negatively, assuming "SETH", as follows.
\begin{theorem}\label{theorem: FO3 model checking hardness}
    Under "SETH",
    for any $k \ge 1$ and any $\eps > 0$,\footnote{When $k \le 2$,
    the "model checking problem" cannot be decided in $\poly(\len{\fml}) \cdot n^{k - \eps}$ time (even when $\fml$ is fixed), as the input adjacent matrix is given by $\Theta(n^2)$ cells (\Cref{proposition: trivial lowerbound}, for a more precise proof).}
    the "model checking problem" for $\fo^{k}$ "sentences" cannot be decided in $\poly(\len{\fml}) \cdot n^{k - \eps}$ time.
\end{theorem}
\Cref{theorem: FO3 model checking hardness} is a direct consequence of the following lemma.
\begin{lemma}\label{lemma: FO3 model checking hardness}
    Suppose that there are an integer $k \ge 1$, a function $f$, and an $\eps > 0$ such that
    the "model checking problem" for $\fo^{k}$ "sentences" is solvable in $\mathcal{O}(f(\len{\fml}) \cdot n^{k - \eps})$ time.
    Then "CNF"-"SAT" is in $\mathcal{O}(f(\len{\fml}) \cdot 2^{n (1 - \eps/k)})$ time,
    where $\len{\fml}$ is the "length@@propositional" of the "CNF" $\fml$ and $n$ is the number of "variables@@propositional" in $\fml$.
\end{lemma}
\begin{proof}
    Let $\fml$ be a "CNF" with $n$ "variables@@propositional".
    "Wlog", assume that $k$ divides $n$.
    Let $p_1, \dots, p_{n}$ be "variables@@propositional" in $\fml$.
    We define $\struc$ as the "structure" with $2^{n/k}$ "vertices" and "unary relation symbols" $P_1, \dots, P_{n/k}$ such that 
    for every distinct pair $\tuple{v, w}$ of "vertices" in $\struc$,
    there is some "unary relation symbol" $P_i$ such that
    $v \in P_i^{\struc}$ and $w \not\in P_i^{\struc}$ hold or 
    $v \not\in P_i^{\struc}$ and $w \in P_i^{\struc}$ hold.
    We also define $\fml[2]$ as the following $\fo^{k}$ "sentence":
    \[\exists z_1. \dots \exists z_k. \fml[1]',\]
    where $\fml[1]'$ is the $\fml$ in which
    each "variable@@propositional" $p_{kq+r}$ (where $0 \le q < n/k$ and $1 \le r \le k$) has been replaced with the "atomic" "formula" $P_q(z_r)$.
    For instance, when $\fml$ is the following "CNF" with $6$ "variables@@propositional" $(p_1 \lor p_3) \land (p_5 \lor p_2) \land (p_4 \lor p_6)$,
    the $\fo^{3}$ "sentence" $\fml[2]$ is given as follows:
    \[\exists z_1.\exists z_2.\exists z_3. (P_1(z_1) \lor P_1(z_3)) \land (P_2(z_2) \lor P_1(z_2)) \land (P_2(z_1) \lor P_2(z_3)).\]
    For each "valuation" $\mathfrak{v}$ of $\fml$, 
    let $\mathfrak{v}'$ be the (unique) "valuation" so that for all $0 \le q < n/k$ and $1 \le r \le k$,
    $\mathfrak{v}(p_{kq+r})$ is true "iff" $\mathfrak{v}'(z_r) \in P_q^{\struc}$.
    Then, $\fml$ is true at $\mathfrak{v}$ "iff" $\struc, \mathfrak{v}' \models \fml[1]'$.
    As the map above is a bijection, we have that $\fml$ is "satisfiable" "iff" $\struc \models \fml[2]$.
    Thus, by applying the assumption,
    "CNF"-"SAT" is in $\mathcal{O}(f(\len{\fml}) \cdot (2^{n/k})^{k-\eps})$ time.
\end{proof}

\begin{proof}[Another proof of \Cref{theorem: FO3 model checking hardness}]
    TODO:
    \begin{yoshiki}
    \Ja{基本方針は同じだが、unary relation symbols を変数でなく節に対応づけてもよい。}\\
    \[\exists z_1. \dots \exists z_k. \bigwedge_{i = 1}^{m} \bigvee_{j = 1}^{k} C_i(z_j).\]
    \Ja{(MAX-2SAT to MAX triangle の帰着 \cite{williamsExactAlgorithmsMaximum2016} を $2^{m}$次元$(0,1)$ベクトルの重みで考えている)}

    \Ja{式の長さ:$\mathcal{O}(k m)$, モデルの頂点数:$\mathcal{O}(k 2^{n/k})$}

    \Ja{(\Cref{theorem: FO3 model checking hardness} の帰着)}
    \Ja{式の長さ:$\mathcal{O}(\len{\fml})$, モデルの頂点数:$\mathcal{O}(2^{n/k})$}
    \end{yoshiki}
\end{proof}


\begin{yoshiki}
    TODO: Add a note on succinctness.
\end{yoshiki}

% Furthermore,
% \Cref{theorem: FO3 model checking hardness} also shows that there is no polynomial-time translation from $\fo^{3}$ "sentences" to "atomic" "CoR" "formulas".
% \begin{corollary}\label{corollary: succinceness under SETH}
%     Under "SETH",
%     there is no polynomial-time translation from $\fo^{3}$ "sentences" to "atomic" "CoR" "formulas".
% \end{corollary}
% \begin{proof}
%     If there is a polynomial-time translation from $\fo^{3}$ "sentences" to "CoR" "formulas",
%     then by \Cref{proposition: CoR model checking}, the "model checking problem" for $\fo^{k}$ "sentences" is solvable in $\poly(\len{\fml}) \cdot n^{\omega}$ time,
%     but this contradicts \Cref{theorem: FO3 model checking hardness} (as $\omega < 3$).
% \end{proof}
% Similar to \Cref{corollary: succinceness under SETH},
% the following question is raised in \cite{nakamuraExpressivePowerSuccinctness2022}.
% \begin{conjecture}[\cite{nakamuraExpressivePowerSuccinctness2022}]\label{conjecture: succinceness CoR}
%     There is no polynomial-size translation from $\fo^{3}$ "sentences" to "atomic" "CoR" "formulas".
% \end{conjecture}

Moreover, the result above still holds even over the "signature" with exactly one binary relation symbol.
\begin{sideyoshiki}
\Cref{theorem: FO3 model checking hardness 1 binary} uses non-"PNF" "formulas".
\end{sideyoshiki}
\begin{theorem}\label{theorem: FO3 model checking hardness 1 binary}
    Under "SETH",
    for any $k \ge 1$ and any $\eps > 0$,\footnote{When $k \le 2$,
    the "model checking problem" cannot be decided in $\poly(\len{\fml}) \cdot n^{k - \eps}$ time (even when $\fml$ is fixed), as the input adjacent matrix is given by $\Theta(n^2)$ cells (\Cref{proposition: trivial lowerbound}, for a more precise proof).}
    the "model checking problem" for $\fo^{k}$ "sentences" with exactly one binary relation symbol $E$ cannot be decided in $\poly(\len{\fml}) \cdot n^{k - \eps}$ time.
\end{theorem}
\Cref{theorem: FO3 model checking hardness 1 binary} is a direct consequence of the following lemma.
\begin{lemma}\label{lemma: FO3 model checking hardness 1 binary}
    Suppose that there is an integer $k \ge 2$ and an $\eps > 0$ such that
    the "model checking problem" for $\fo^{k}$ "sentences" with exactly one binary relation symbol $E$ is solvable in $\poly(\len{\fml}) \cdot n^{k - \eps}$ time. 
    Then "CNF"-"SAT" is in $\poly(\len{\fml}) \cdot 2^{n (1 - \eps/k)}$ time,
    where $\len{\fml}$ is the "length@@propositional" of the "CNF" $\fml$ and $n$ is the number of "variables@@propositional" in $\fml$.
\end{lemma}
\begin{proof}
    Let $\fml$ be a "CNF" with $n$ "variables@@propositional".
    "Wlog", assume that $k$ divides $n$.
    Let $p_1, \dots, p_{n}$ be "variables@@propositional" in $\fml$.
    We define the "structure" $\struc$ as follows:
    \begin{itemize}
        \item $\univ{\struc} \defeq (\set{\mathrm{V}} \times \range{0, 2^{n/k}-1}) \uplus (\set{\mathrm{E}} \times \range{1, n/k})$; we write $v^{\mathrm{V}}$ for $\tuple{\mathrm{V}, v}$ (similarly for $v^{\mathrm{E}}$).
        \item $E^{\struc} \defeq \set{\tuple{v^{\mathrm{V}}, w^{\mathrm{E}}} \mid v \in \range{0, 2^{n/k}-1}, w \in \range{1, n/k}, \text{ and the $w$-th bit of $v$ is 1}} \cup \set{\tuple{w^{\mathrm{E}}, (w+1)^{\mathrm{E}}} \mid w \in \range{1, n-1}} \cup \set{\tuple{n^{\mathrm{E}}, n^{\mathrm{E}}}}$.
    \end{itemize}
    We also define $\fml[2]$ as the following $\fo^{k}$ "sentence":
    \[\exists z_1. \dots \exists z_k. \fml[1]',\]
    where $\fml[1]'$ is the $\fml$ in which
    each "variable@@propositional" $p_{kq+r}$ (where $0 \le q < n/k$ and $1 \le r \le k$) has been replaced with
    the  $\fo^{2}$  "formula" $X_q(z_r)$ defined as follows:
    \[X_q(z) \defeq \begin{cases}
        E(z, z) & \text{if $q = 0$,}\\
        \exists z'. E(z, z') \land X_{q-1}(z') & \text{if $q > 0$ where $z'$ is a "variable" not equal to $z$}.
    \end{cases}\]
    Note that $X_q(z_r)$ is "semantically equivalent" to the "CoR" "formula" $(E^{q} (E \cap \id))^{\dom}(z_r, z_r)$.

    For instance when $k = 3$, the "variable@@propositional" $p_{2 \cdot 1 + 3}$ is 
    transformed to the following "formula":
    \[\exists z'. E(z_3, z') \land (\exists z''. E(z', z'') \land E(z'', z'')).\]
    For each "valuation@@propositional" $\mathfrak{v}$ of $\fml$, 
    we can take a "valuation" $\mathfrak{v}'$ of $\fml'$ so that for all $0 \le q < n/k$ and $1 \le r \le k$,
    $\mathfrak{v}(p_{kq+r})$ is true "iff" $\struc, \mathfrak{v}' \models X_q(z_r)$.
    Then, $\fml$ is true at $\mathfrak{v}$ "iff" $\struc, \mathfrak{v}' \models \fml[1]'$.
    Conversely, for each "valuation" $\mathfrak{v}'$ of $\fml'$, we can take a "valuation@@propositional" $\mathfrak{v}$ of $\fml$ so that
    $\fml$ is true at $\mathfrak{v}$ "iff" $\struc, \mathfrak{v}' \models \fml[1]'$.
    Thus, we have that $\fml$ is "satisfiable" "iff" $\struc \models \fml[2]$.
    Thus, by applying the assumption,
    "CNF"-"SAT" is in $\poly(n/k \cdot \len{\fml}) \cdot (2^{n/k} + n/k)^{k-\eps}$ time.%
    As $n \le \len{\fml}$, this implies $\poly(\len{\fml}) \cdot 2^{n(1-\eps/k)}$.
\end{proof}

\begin{yoshiki}
    TODO: Add a dichotomy (\Ja{メモ \ref{memo: dichotomy}}) in some section.
\end{yoshiki}

% \begin{yoshiki}
%     TODO: Add a remark on the reduction using "$k$-dominating set" \cite{patrascuPossibilityFasterSAT2010} (?) 
% \end{yoshiki}

% One approach is based on to use the reduction from "CNF-SAT" to "$4$-dominating set" of \cite[Corollary 1.4]{williamsFasterDecisionFirstorder2014}.
% \begin{proposition}\label{proposition: SETH dominating set}
%     If the "model checking problem" for $\fo^{k}$ \kl{sentences} can be decided in $\poly(\len{\fml}) \cdot n^{k - 1 - \varepsilon}$ time for some $k \ge 4$ and some $\varepsilon > 0$,\footnote{When $k \le 3$,
%     the "model checking problem" cannot be decided in $\poly(\len{\fml}) \cdot n^{k - \varepsilon}$ time (even when $\fml$ is fixed), as the input adjacent matrix is given by $\Theta(n^2)$ cells (\Cref{proposition: trivial lowerbound}, for a more precise proof).}
%     then "SETH" is false.
% \end{proposition}
% \begin{proof}
%     By \cite[Theorem 2.1]{patrascuPossibilityFasterSAT2010}\cite[Corollary 1.4]{williamsFasterDecisionFirstorder2014},
%     the "model checking problem" for $\fo^{k}$ \kl{sentences} can be reduced to the "$k$-dominating set" problem.
% \end{proof}

\section{Conclusion}\label{sec: conclusion}


\bibliography{../../../../library.bib} % for yn

\end{document}
