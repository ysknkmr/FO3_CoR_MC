\section{Preliminaries}\label{sec: preliminaries}

% \section{Notation}

% \begin{hidden}
% \Ja{
% (メモ:
% LIPICSとpLaTeXは異常に組み合わせが悪いので、
% コメントなどで日本語を書く時はこのようにします。)}

% \begin{screen}
% Cute round-corner box \\
% \Ja{コメントを付ける時には、screenで囲うとこういう感じになります}
% \end{screen}

% \begin{yoshiki}
%   \Ja{コメントテスト}
% \end{yoshiki}

% \begin{sideyoshiki}
%   \Ja{コメントテスト2}
% \end{sideyoshiki}
% \end{hidden}

\AP We write $\Z$ for the set of all integers.

\AP We write $\range{l, r}$ for the set $\set{i \in \Z \mid l \le i \le r}$.

\AP A ""relational signature"" (henceforth, "signature") $\sig$ is ...

\AP A ""structure"" $\struc$ over a "signature" $\sig$ is a tuple $\tuple{\univ{\struc}, \set{R^{\struc}}_{R \in \sig}}$,
where its ""universe"" $\univ{\struc}$ is a finite\footnote{In this paper, we are only interested in finite "structures".} set of ""vertices""
and each $R^{\struc}$ is a binary relation on $\univ{\struc}$.

\AP A ""sentence"" is a "formula" without free "variables@@propositional".

For a "formula" $\fml$ and a "valuation" $\mathfrak{v}$, we write 
$\struc, \mathfrak{v} \models \fml$ to denote that $\fml$ is true at $\struc$ under $\mathfrak{v}$.
For a "sentence" $\fml$, we write $\struc \models \fml$ to denote that $\fml$ is true at $\struc$.

We write $\fo^k$ to denote the set of formulas of First-order predicate logic with $k$-variables.

\paragraph*{Important Remark (Unary and Binary Predicates)}
We allow unary and binary predicates in our $\fo^3$.
Therefore, we do not allow terms involving ternary predicates such as $\forall x.\,\exists y.\,\forall z.\, P(x, y, z)$.%
\footnote{For each unary predicate $U$, we can simulate it by a diagonal binary predicate $B_U(x, y) := x = y \land U(x)$. However, in our later construction, we use some unary predicates; thus, we also allow unary predicates explicitly.}

On this remark, we can identify each model $\mathcal{M}$ as color graphs as follows:
\begin{itemize}
\item Each element corresponds a node (vertex).
\item Each unary predicate $U_i$ means ``color-$i$'' node.
\item Each binary predicate $P_j$ is ``color-$j$'' edge.
\item For example, on two nodes $x, y$, a term $(U_i(x) \land U_j(y) \land P_k(x, y))$ means that $x$ has the color $i$, $y$ has the color $j$, and there is a color $k$ edge from $x$ to $y$.
\end{itemize}

\paragraph*{Quantifier width}
\begin{Jcomment}
Quantifier widthという謎の指標を導入します。

与えられた式の部分項 $\forall/\exists x. ( \cdots )$ について $\cdots$部分に「直接」出てくるquantifierの個数を、この部分項のquantifier widthと呼ぶことにします。

たとえば、次の部分項
$$
\forall x.\biggl( \Bigl(\underline{{\color{red}\forall}} y. P(y)\Bigr) \lor \Bigl(\underline{{\color{red}\forall}} z. (\underline{\forall} x. P(z, x)) \land Q(z)\Bigr) \biggr)
$$
については、内部に3つのquantifierの出現(下線)があるのですが、直接の出現は赤くした2つだけなので、この部分項のquantifier widthは「2」です。

そして、式のquantifier widthというのは、最大のquantifier widthを持つ部分項における、その数として定義します。
\end{Jcomment}

\paragraph*{Tarski's Calculus of Relations (CoR)}
"CoR" "terms" are generated by the following grammar:
\begin{align*}
    \term[1], \term[2] \;\Coloneqq\;
    & a \mid \term[1] \union \term[2] \mid \term[1] \intersection \term[2] \mid \term[1]^{-}\\
    & \phantom{a} \mid \term[1] \compo \term[2] \mid \term[1] \dagger \term[2] \mid \term[1]^{\pi}.  
\end{align*}

\begin{proposition}\label{proposition: CoR model checking}
    The "model checking problem" for "atomic" "CoR" "formulas" is in $\mathcal{O}(\len{\fml} \cdot n^{\omega})$ time.
\end{proposition}
\begin{proof}[Proof Sketch]
    By a naive bottom-up evaluation algorithm.
    Here, we use the "matrix multiplication" algorithm for the "relational composition" ($\compo$).
\end{proof}

\paragraph*{SETH}

\begin{conjecture}[The ""Strong Exponential Time Hypothesis"" (""SETH"") \cite{impagliazzoComplexity$k$SAT2001,impagliazzoWhichProblemsHave2001}]\label{conjecture: SETH}
    For every $\delta < 1$,
    there exists an integer $k$ such that
    "$k$-CNF"-"SAT" with $n$ "variables@@propositional" cannot be solved in $\mathcal{O}(2^{\delta n})$ time.%
    \begin{sideyoshiki}
    \Ja{たとえば \cite{vassilevskawilliamsHardnessEasyProblems2015} では、$\mathcal{O}(2^{\delta n})$ でなく $2^{\delta n} \poly (n)$ を用いているが、}\\
    \Ja{予想としては同値(のはず。$\poly(n) < \mathcal{O}(2^{\delta n})$ なので)。}\\
    \Ja{$\mathcal{O}(2^{n} \poly(n))$ のアルゴリズムが存在するという fact に合わせている?} 
    \end{sideyoshiki}
\end{conjecture}
""SETH"" implies the following hypothesis.
\begin{sideyoshiki}\label{conjecture: CNF-SAT}
    \Ja{However, it is not known whether the two conjectures are equivalent.}
    \cite[Previous Work]{cyganProblemsHardCNFSAT2016}
\end{sideyoshiki}
\begin{conjecture}["eg", in {\cite{patrascuPossibilityFasterSAT2010}}]
    For every $\delta < 1$,
    "CNF"-"SAT" with $n$ "variables@@propositional" and $m$ "clauses" cannot be solved in $2^{\delta n} \cdot \poly(m)$ time.
\end{conjecture}

