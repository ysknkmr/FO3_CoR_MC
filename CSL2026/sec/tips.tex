
\section{Tips}

\subsection{Trivial Results}

\begin{proposition}[Almost trivial fact]\label{proposition: trivial lowerbound}
  Let $\fml_0$ be the following $\fo^{k}$ "sentence":
  \[\exists x_1. \dots \exists x_k. R(x_1, \dots, x_k).\]
  Then for any $\eps > 0$,
  the "model checking problem" where the "sentence" is fixed to $\fml_0$
  cannot be decided in $\mathcal{O}(n^{k - \eps})$ time (without any assumptions).
\end{proposition}
\begin{proof}
  (Below, we suppose that we can access to at most one cell in each step.)
  Towards a contradiction, 
  assume that for some $M$,
  there is an $M n^{k - \eps}$ time RAM model $\mathcal{M}$ solving the "model checking problem" (for sufficiently large $n$).
  Then, a sufficiently large $n_0 > n$ satisfies $M n_0^{k - \eps} < n_0^{k}$.

  Let $\struc$ be the \kl{structure} with $n_0$ \kl{vertices} such that $R^{\struc} = \emptyset$.
  We consider the run $\rho$ of $\mathcal{M}$ when the input is $\struc$.
  In the run $\rho$ (of time at most $M n^{k - \eps}$), as $M n_0^{k - \eps} < n_0^{k}$,
  there is an input cell $c$ for asserting $\tuple{v_1, \dots, v_k} \in R^{\struc}$ such that $\rho$ does not access $c$.
  We then define $\struc[2]$ as the \kl{structure} $\struc$ in which $R^{\struc[2]}$ has been replaced with the singleton set $\set{\tuple{v_1, \dots, v_k}}$.
  When $\struc[2]$ is input, the run is the same as $\rho$, as the cell $c$ is not accessed in $\rho$.
  Hence, in $\mathcal{M}$, the output of $\struc[2]$ is also the same as that of $\struc$, reaching a contradiction.
  (When $\struc[1]$ is input, the output should be ``no''. When $\struc[2]$ is input, the output should be ``yes''.)
\end{proof}

\begin{yoshiki}
  \Ja{同様の議論がある論文/本からの引用などで簡単に済ませたい。}\\

  \Ja{とくに3項関係記号をもつ $\fo^{3}$ は、$\mathcal{O}(n^{3 - \eps})$ で解けない(行列積で解けるのは2項関係記号の時)。}
\end{yoshiki}

\begin{proposition}\label{proposition: unary}
  The "model checking problem" for $\fo^{k}$ with only "unary relation symbols" (without "equality")
  is solvable in $\mathcal{O}(n)$ time.
\end{proposition}
\begin{proof}[Proof Sketch]
  In this case, by taking the "DNF", we can transform each "subformula" of the form $\exists x. \fml$ into the following form: $\exists x. \bigwedge_{i} P_i(x)$.
  Thus, each "subformula" has at most one "free variable".
  Hence, by the naive evaluation algorithm, we have obtained $\mathcal{O}(n)$ time.
\end{proof}

\subsection{Summary}
\begin{center}
  \begin{tabular}{c||c||c||c}
    Problem / max. arity $k$ & 1 & 2 & 3\\
    \hline
    \multirow{2}{*}{\begin{tabular}{c}
    "PNF" $\fo^{3}$ (data)\\
    $\fo^{3}$ (data)
    \end{tabular}} & $\mathcal{O}(n)$ (\Cref{proposition: unary}) & $\mathcal{O}(n^{\omega})$ \cite{williamsFasterDecisionFirstorder2014} & $\mathcal{O}(n^{3})$ (trivial)\\  
    \cline{2-4}
    & \multicolumn{3}{c}{no $\mathcal{O}(n^{k-\eps})$ (\Cref{proposition: trivial lowerbound})}\\
    \hline
    \multirow{2}{*}{\begin{tabular}{c}
    "PNF" $\fo^{3}$ (combined)\\
    $\fo^{3}$ (combined)
    \end{tabular}} 
    & \begin{tabular}{c}
        $\mathcal{O}(\len{\fml} \cdot n^{3})$ (trivial) \\
        $\mathcal{O}(2^{\len{\fml}} \cdot n)$ (\Cref{proposition: unary})
    \end{tabular}
    &  \begin{tabular}{c}
    $\mathcal{O}(\len{\fml} \cdot n^{3})$ (trivial)\\
    $\mathcal{O}(2^{\len{\fml}} \cdot n^{\omega})$ \cite{williamsFasterDecisionFirstorder2014,nakamuraExpressivePowerSuccinctness2022}
    \end{tabular}
    & $\mathcal{O}(\len{\fml} \cdot n^{3})$ (trivial)\\
    \cline{2-4}
    & \multicolumn{3}{c}{no $\poly(\len{\fml}) \cdot n^{3 - \eps}$, under "SETH" (\Cref{theorem: FO3 model checking hardness})}\\  
  \end{tabular}
\end{center}

\Ja{(以下おまけ)}
\begin{center}
\scalebox{.7}{
  \begin{tabular}{c||c||c|c||c}
    Problem / max. arity $k$ & 1 & 2 & 3 & 4\\
    \hline
    \multirow{2}{*}{\begin{tabular}{c}
    "PNF" $\fo^{4}$ (data)\\
    $\fo^{4}$ (data)
    \end{tabular}} & $\mathcal{O}(n)$ (\Cref{proposition: unary}) & 
    \begin{tabular}{c}
        \textcolor{red}{$\mathcal{O}(n^{1 + \omega})$} \cite[Cor.\ 1.3]{williamsFasterDecisionFirstorder2014} (?? for $\fo^{4}$) \\
        \textcolor{red}{$\mathcal{O}(n^{\omega_3})$} \\
    \end{tabular}
     & \textcolor{red}{$\mathcal{O}(n^{\omega_3})$} & $\mathcal{O}(n^{4})$ (trivial) \\  
    \cline{2-5}
    & no $\mathcal{O}(n^{1-\eps})$ (\Cref{proposition: trivial lowerbound}) &
    no $\mathcal{O}(n^{3-\eps})$ under "SETH" \cite[Cor.\ 1.4]{williamsFasterDecisionFirstorder2014} &
    \multicolumn{2}{c}{no $\mathcal{O}(n^{k-\eps})$ (\Cref{proposition: trivial lowerbound})}\\
    \hline
    \multirow{2}{*}{\begin{tabular}{c}
    "PNF" $\fo^{4}$ (combined)\\
    $\fo^{4}$ (combined)
    \end{tabular}}
    & \begin{tabular}{c}
        $\mathcal{O}(\len{\fml} \cdot n^{4})$ (trivial)\\
        $\mathcal{O}(2^{\len{\fml}} \cdot n)$ (\Cref{proposition: unary})
    \end{tabular}
    & \begin{tabular}{c}
        $\mathcal{O}(\len{\fml} \cdot n^{4})$ (trivial)\\
        \textcolor{red}{$\mathcal{O}(2^{\len{\fml}} \cdot n^{1 + \omega})$} \cite[Cor.\ 1.3]{williamsFasterDecisionFirstorder2014} (?? for $\fo^{4}$)\\
        \textcolor{red}{$\mathcal{O}(2^{\len{\fml}} \cdot n^{\omega_3})$}
    \end{tabular}
    & \begin{tabular}{c}
        $\mathcal{O}(\len{\fml} \cdot n^{4})$ (trivial)\\
        \textcolor{red}{$\mathcal{O}(2^{\len{\fml}} \cdot n^{\omega_3})$}
    \end{tabular}
    & $\mathcal{O}(\len{\fml} \cdot n^{4})$ (trivial)\\
    \cline{2-5}
    & \multicolumn{4}{c}{no $\poly(\len{\fml}) \cdot n^{4 - \eps}$, under "SETH" (\Cref{theorem: FO3 model checking hardness})}\\  
  \end{tabular}
}
\end{center}


\begin{center}
\scalebox{.5}{
  \begin{tabular}{c||c||c|c|c||c}
    Problem / max. arity $k$ & 1 & 2 & 3 & 4 & 5\\
    \hline
    \multirow{2}{*}{\begin{tabular}{c}
    "PNF" $\fo^{5}$ (data)\\
    $\fo^{5}$ (data)
    \end{tabular}} & $\mathcal{O}(n)$ (\Cref{proposition: unary}) & 
    \begin{tabular}{c}
        \textcolor{red}{$\mathcal{O}(n^{2 + \omega})$} \cite[Cor.\ 1.3]{williamsFasterDecisionFirstorder2014} (?? for $\fo^{5}$) \\
        \textcolor{red}{$\mathcal{O}(n^{1 + \omega_3})$} (?? for $\fo^{5}$) \\
        \textcolor{red}{$\mathcal{O}(n^{\omega_4})$} \\
    \end{tabular}
    & \begin{tabular}{c}
        \textcolor{red}{$\mathcal{O}(n^{1 + \omega_3})$} \cite[Cor.\ 1.3]{williamsFasterDecisionFirstorder2014} (?? for $\fo^{5}$) \\
        \textcolor{red}{$\mathcal{O}(n^{\omega_4})$} \\
    \end{tabular} & \textcolor{red}{$\mathcal{O}(n^{\omega_4})$} & $\mathcal{O}(n^{5})$ (trivial) \\  
    \cline{2-6}
    & no $\mathcal{O}(n^{1-\eps})$ (\Cref{proposition: trivial lowerbound}) &
    \multicolumn{2}{c|}{no $\mathcal{O}(n^{4-\eps})$ under "SETH" \cite[Cor.\ 1.4]{williamsFasterDecisionFirstorder2014}} &
    \multicolumn{2}{c}{no $\mathcal{O}(n^{k-\eps})$ (\Cref{proposition: trivial lowerbound})}\\
    \hline
    \multirow{2}{*}{\begin{tabular}{c}
    "PNF" $\fo^{5}$ (combined)\\
    $\fo^{5}$ (combined)
    \end{tabular}}
    & \begin{tabular}{c}
        $\mathcal{O}(\len{\fml} \cdot n^{5})$ (trivial)\\
        $\mathcal{O}(2^{\len{\fml}} \cdot n)$ (\Cref{proposition: unary})
    \end{tabular}
    & \begin{tabular}{c}
        $\mathcal{O}(\len{\fml} \cdot n^{5})$ (trivial)\\
        \textcolor{red}{$\mathcal{O}(2^{\len{\fml}} \cdot n^{2 + \omega})$} \cite[Cor.\ 1.3]{williamsFasterDecisionFirstorder2014} (?? for $\fo^{5}$)\\
        \textcolor{red}{$\mathcal{O}(2^{\len{\fml}} \cdot n^{1 + \omega_3})$} (?? for $\fo^{5}$)\\
        \textcolor{red}{$\mathcal{O}(2^{\len{\fml}} \cdot n^{\omega_4})$}
    \end{tabular}
    & \begin{tabular}{c}
        $\mathcal{O}(\len{\fml} \cdot n^{5})$ (trivial)\\
        \textcolor{red}{$\mathcal{O}(n^{1 + \omega_3})$} (?? for $\fo^{5}$)\\
        \textcolor{red}{$\mathcal{O}(2^{\len{\fml}} \cdot n^{\omega_4})$}
    \end{tabular}
    & 
    \begin{tabular}{c}
        $\mathcal{O}(\len{\fml} \cdot n^{5})$ (trivial)\\
        \textcolor{red}{$\mathcal{O}(2^{\len{\fml}} \cdot n^{\omega_4})$}
    \end{tabular}
    &
    \begin{tabular}{c}
        $\mathcal{O}(\len{\fml} \cdot n^{5})$ (trivial)\\
    \end{tabular}\\
    \cline{2-6}
    & \multicolumn{5}{c}{no $\poly(\len{\fml}) \cdot n^{5 - \eps}$, under "SETH" (\Cref{theorem: FO3 model checking hardness})}\\ 
  \end{tabular}
}
\end{center}

\begin{yoshiki}
\begin{itemize}
    \item \Ja{?? は主張が不明/怪しい箇所(メモ\ref{memo: FO4}参照)}
    \begin{itemize}
        \item \Ja{主張が成り立たなそうであることを言うには、}\\
        \Ja{Binary $\fo^{4}$ is in $\mathcal{O}(n^{4 - \eps})$ time $\Longrightarrow$ $\omega_{3} < 4$ を言えればよさそう。}
        \begin{observation}
            \begin{itemize}
                \item $\omega_2 = \omega < 2.38 (< 3)$.\\
                \item $\omega_{3} < 4$ is wished \cite{williamsFasterDecisionFirstorder2014} but open.\\
                \Ja{4-clique in a 3-uniform hypergraph を含む難しい問題のようです。Cf.\ HyperClique Conjecture}
            \end{itemize}
        \end{observation}

        \item \Ja{もしくは、もう少し精密化して今回の設定に合った行列積の問題に一般化する。}

        \Ja{入力を$\range{n}$上の任意有限個の$m$項関係として、$k$引数関係 $A_1, \dots, A_{k}$ は入力の$m$項関係から $\fo^{k+1}$-definableであるとする。この時の``一般化行列積''}
        \[D[i_1, \dots, i_k] = \bigvee_{\ell = 1}^{n} \bigwedge_{j = 1}^{k} A_j [i_1, \dots i_{j-1}, i_{j+1}, \dots, i_{k}, \ell]\]
        \Ja{の時間計算量を $\mathcal{O}(n^{\omega_{k, m}})$ とする。}

        \begin{itemize}
            \item \Ja{Fact: $\omega_{k,k} = \omega_{k}$。とくに $\omega_{2,2} = \omega$。\\
            ($\omega_{k, k} \le \omega_{k}$: $\fo^{k+1}$ "model checking" が $\mathcal{O}(n^{\omega_{k}})$より。 $\omega_{k, k} \ge \omega_{k}$: $A_1, \dots, A_k$ が入力になる場合を考える)}
            \item \Ja{Fact: $\omega_{k,1} = k$ (成分ごと独立なので)}
            \item \Ja{Fact: $k = \omega_{k, 1} \le \omega_{k,2} \le \dots \le \omega_{k,k} = \omega_{k} \le k+1$}
            \item \Ja{Fact: $\omega_{k} = k+1$, under "$k$-uniform HyperClique Hypothesis" \cite{bringmannFineGrainedAnalogueSchaefers2019}, for $k \ge 3$}
        \end{itemize}

        $\leadsto$
        \Ja{(??) $\omega_{k,2} = k+1$, under ``Strong'' "$k$-uniform HyperClique Hypothesis", for $k \ge 3$}\\
        \Ja{みたいなことを言っておいても許される??}        
        
        We then have:
        \begin{proposition}
            Binary $\fo^{4}$ is in $\mathcal{O}(n^{c})$ time $\iff$ $\omega_{3,2} \le c$.
        \end{proposition}
        \begin{proof}
            $\Longleftarrow$: By an analog of the FO3-to-CoR translation.
            $\Longrightarrow$: 
            Every matrix multiplication problem "wrt" $\omega_{3,2}$ can 
            be expressed as a "model checking problem" where the "sentence" is fixed to a binary $\fo^{4}$ "sentence".
        \end{proof}

        \begin{corollary}\label{corollary: omega 3 2}
            If binary $\fo^{4}$ is in $\mathcal{O}(n^{1 + \omega})$ time, then $\omega_{3,2} \le 1 + \omega$.
            Hence, $\omega_{3, 2} < 4$.
        \end{corollary}

         \Ja{$\leadsto$ ``Strong'' "$k$-uniform HyperClique Hypothesis" に反する}

         \Ja{$\leadsto$ $\omega_{3, 2}$ の問題の形で書けるが $\mathcal{O}(n^{4-\eps})$ が open な問題はあるでしょうか?}
    \end{itemize}
    \item \Ja{赤の箇所は未証明}
\end{itemize}
\end{yoshiki}

\begin{yoshiki}
  \begin{itemize}
    \item \Ja{"SETH" $\Longrightarrow$ Hyperclique Hypothesis(参照 \href{https://cstheory.stackexchange.com/a/32251}{https://cstheory.stackexchange.com/a/32251})}
    \item \Ja{"SETH" $\Longrightarrow$ ``Strong'' Hyperclique Hypothesis は未検証 (Open Problem)。}\\
    \Ja{もしこれが成り立つと \cite[Theorem 7]{gaoFineGrainedComplexityStrengthenings2019} は誤り(under "SETH")。}
  \end{itemize} 
\end{yoshiki}


% \subsection{An Efficient "CNF"-"SAT"}
% We say that a "CNF" with $n$ \kl(propositional){variables} is ""$k$-decomposable"" if
% there exists a $k$-partition $V_1, \dots, V_k$ of the \kl(propositional){variables} 
% such that
% \begin{itemize}
%   \item $\#V_i = n/k$,
%   \item for each "clause", there is some $V_i$ such that the "clause" has no \kl(propositional){variables} in $V_i$.
% \end{itemize}
% ""$k$-decomposed-CNF-SAT"" is defined as the following problem:
% given a "$k$-decomposable" "CNF" $\fml$ with its $k$-partition $V_1, \dots V_k$ witness for "$k$-decomposable",
% is the "CNF" $\fml$ "satisfiable"?

% \begin{proposition}\label{proposition: 3-decomposed-CNF-SAT}
%   "$k$-decomposed-CNF-SAT" is in $\mathcal{O}(2^{\omega_{k-1}/k \cdot n} \cdot m)$ time,
%   where $\omega_{k-1}$ is the exponent of $(k-1)$-dimension matrix multiplication.\begin{sideyoshiki}\Ja{メモ\ref{memo: FO4} 参照}\end{sideyoshiki}
% \end{proposition}
% \begin{proof}[Proof Sketch]
%   This reduction is inspired by the reduction from "CNF"-"SAT" to the "$k$-dominating set" \cite{patrascuPossibilityFasterSAT2010}.
%   The "$k$-dominating set" is expressed as the $\fo^{k+1}$ "model checking problem" where 
%   the "sentence" is fixed as follows:
%   \[\exists z_1. \dots \exists z_k. \forall w. \bigvee_{i = 1}^{k} E(z_i, w).\]
%   In particular for "$k$-decomposed"-"CNF",
%   we can reduce into the $\fo^{k}$ "model checking problem"
%   where the "sentence" is fixed as follows:
%   \begin{align*}
%     &\exists z_1. \dots. \exists z_k. \bigwedge_{i = 1}^{k} (\forall z_i. P_{V_i}(z_i) \to \bigvee_{j \in \range{1, k}; j \neq i}E(z_j, z_i)).
%   \end{align*}
%   Here, $P_{V_i}(z_i)$ expresses that the "clause" indicated by $z_i$ has no \kl(propositional){variables} in $V_i$.
% \end{proof}

% \begin{yoshiki}
%   \Ja{"CNF"-"SAT" が $2^{\delta n} \poly(m)$ で解けないという仮説 (\Cref{conjecture: CNF-SAT})}\\
%   \Ja{に対して、"$3$-decomposed" ならば $2^{\delta n} \poly(m)$ で解ける}\\
%   \Ja{一般に $\omega_{k-1} < k$ の、"$k$-decomposed" に対して $2^{\delta n} \poly(m)$ で解ける}

%   \Ja{(メモ \ref{memo: FO4} の hardness($\mathcal{O}(n^{k-\eps})$の非存在)側を考えていた時の失敗作です。 $V_1, \dots, V_k$ を与える必要があるので、そんなに面白みはないのかもしれません)}
% \end{yoshiki}