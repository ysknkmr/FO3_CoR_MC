
\section{Tips}

\subsection{Trivial Results}

\begin{proposition}[Almost trivial fact]\label{proposition: trivial lowerbound}
  Let $\fml_0$ be the following $\fo^{k}$ "sentence":
  \[\exists x_1. \dots \exists x_k. R(x_1, \dots, x_k).\]
  Then for any $\eps > 0$,
  the "model checking problem" where the "sentence" is fixed to $\fml_0$
  cannot be decided in $\mathcal{O}(n^{k - \eps})$ time (without any assumptions).
\end{proposition}
\begin{proof}
  (Below, we suppose that we can access to at most one cell in each step.)
  Towards a contradiction, 
  assume that for some $M$,
  there is an $M n^{k - \eps}$ time RAM model $\mathcal{M}$ solving the "model checking problem" (for sufficiently large $n$).
  Then, a sufficiently large $n_0 > n$ satisfies $M n_0^{k - \eps} < n_0^{k}$.

  Let $\struc$ be the \kl{structure} with $n_0$ \kl{vertices} such that $R^{\struc} = \emptyset$.
  We consider the run $\rho$ of $\mathcal{M}$ when the input is $\struc$.
  In the run $\rho$ (of time at most $M n^{k - \eps}$), as $M n_0^{k - \eps} < n_0^{k}$,
  there is an input cell $c$ for asserting $\tuple{v_1, \dots, v_k} \in R^{\struc}$ such that $\rho$ does not access $c$.
  We then define $\struc[2]$ as the \kl{structure} $\struc$ in which $R^{\struc[2]}$ has been replaced with the singleton set $\set{\tuple{v_1, \dots, v_k}}$.
  When $\struc[2]$ is input, the run is the same as $\rho$, as the cell $c$ is not accessed in $\rho$.
  Hence, in $\mathcal{M}$, the output of $\struc[2]$ is also the same as that of $\struc$, reaching a contradiction.
  (When $\struc[1]$ is input, the output should be ``no''. When $\struc[2]$ is input, the output should be ``yes''.)
\end{proof}

\begin{yoshiki}
  \Ja{同様の議論がある論文/本からの引用などで簡単に済ませたい。}\\

  \Ja{とくに3項関係記号をもつ $\fo^{3}$ は、$\mathcal{O}(n^{3 - \eps})$ で解けない(行列積で解けるのは2項関係記号の時)。}
\end{yoshiki}

\begin{proposition}\label{proposition: unary}
  The "model checking problem" for $\fo^{k}$ with only "unary relation symbols" (without "equality")
  is solvable in $\mathcal{O}(n)$ time.
\end{proposition}
\begin{proof}[Proof Sketch]
  In this case, by taking the "DNF", we can transform each "subformula" of the form $\exists x. \fml$ into the following form: $\exists x. \bigwedge_{i} P_i(x)$.
  Thus, each "subformula" has at most one "free variable".
  Hence, by the naive evaluation algorithm, we have obtained $\mathcal{O}(n)$ time.
\end{proof}

\begin{yoshiki}
\Ja{TODO: unary $+$ equality の場合はどうなるか?}
\end{yoshiki}

\subsection{Current Summary}
\begin{table}[h]
  \begin{tabular}{c||c||c||c}
    Problem / max. arity $k$ & 1 & 2 & 3\\
    \hline
    \multirow{2}{*}{\begin{tabular}{c}
    "PNF" $\fo^{3}$ (data)\\
    $\fo^{3}$ (data)
    \end{tabular}} & $\mathcal{O}(n)$ (\Cref{proposition: unary}) & $\mathcal{O}(n^{\omega})$ \cite{williamsFasterDecisionFirstorder2014} & $\mathcal{O}(n^{3})$ (trivial)\\  
    \cline{2-4}
    & \multicolumn{3}{c}{no $\mathcal{O}(n^{k-\eps})$ (\Cref{proposition: trivial lowerbound})}\\
    \hline
    \multirow{2}{*}{\begin{tabular}{c}
    "PNF" $\fo^{3}$ (combined)\\
    $\fo^{3}$ (combined)
    \end{tabular}} 
    & \begin{tabular}{c}
        $\mathcal{O}(\len{\fml} \cdot n^{3})$ (trivial) \\
        $\mathcal{O}(2^{\len{\fml}} \cdot n)$ (\Cref{proposition: unary})
    \end{tabular}
    &  \begin{tabular}{c}
    $\mathcal{O}(\len{\fml} \cdot n^{3})$ (trivial)\\
    $\mathcal{O}(2^{\len{\fml}} \cdot n^{\omega})$ \cite{williamsFasterDecisionFirstorder2014,nakamuraExpressivePowerSuccinctness2022}
    \end{tabular}
    & $\mathcal{O}(\len{\fml} \cdot n^{3})$ (trivial)\\
    \cline{2-4}
    & \multicolumn{3}{c}{no $\poly(\len{\fml}) \cdot n^{3 - \eps}$ under "SETH" (\Cref{theorem: FO3 model checking hardness})}\\  
  \end{tabular}
  \caption{Complexity for $\fo^{3}$.}
\label{table: FO3}
\end{table}


A similar figure can be found in \cite[Figure 1]{bringmannFineGrainedAnalogueSchaefers2019} for data complexity (sparse graphs).

\clearpage
\section{Hyperclique hypothesis and the case of arity $k \ge 4$}

\begin{table}[h]
\scalebox{.6}{
  \begin{tabular}{c||c||c|c||c}
    / max. arity $k$ & 1 & 2 & 3 & 4\\
    \hline
    \multirow{2}{*}{\begin{tabular}{c}
    "PNF" $\fo^{4}$\\(data)
    \end{tabular}} & $\mathcal{O}(n)$ (\Cref{proposition: unary}) & 
    \begin{tabular}{c}
        $\mathcal{O}(n^{1 + \omega})$ \cite[Cor.\ 1.3]{williamsFasterDecisionFirstorder2014}, \textcolor{red}{$\mathcal{O}(n^{\omega_3})$} \\
    \end{tabular}
    & \textcolor{red}{$\mathcal{O}(n^{\omega_3})$}
    & \multicolumn{1}{c}{$\mathcal{O}(n^{4})$ (trivial)} \\  
    \cline{2-5} &
    no $\mathcal{O}(n^{1-\eps})$ (\Cref{proposition: trivial lowerbound}) &
    no $\mathcal{O}(n^{3-\eps})$ under "SETH" \cite[Cor.\ 1.4]{williamsFasterDecisionFirstorder2014} &
    no $\mathcal{O}(n^{4-\eps})$ under "HyperClique" &
    \multicolumn{1}{c}{no $\mathcal{O}(n^{k-\eps})$ (\Cref{proposition: trivial lowerbound})}\\
    \hline
    \multirow{2}{*}{\begin{tabular}{c}
    $\fo^{4}$\\(data)
    \end{tabular}} & $\mathcal{O}(n)$ (\Cref{proposition: unary}) & 
    \begin{tabular}{c}
        \textcolor{red}{$\mathcal{O}(n^{\omega_3})$} \\
    \end{tabular}
    & \textcolor{red}{$\mathcal{O}(n^{\omega_3})$}
    & \multicolumn{1}{c}{$\mathcal{O}(n^{4})$ (trivial)} \\  
    \cline{2-5} &
    no $\mathcal{O}(n^{1-\eps})$ (\Cref{proposition: trivial lowerbound}) &
    \begin{tabular}{c}
      no $\mathcal{O}(n^{4-\eps})$ under "binary HyperClique" (\Cref{proposition: binary encoded hyperclique hypothesis})\\
      no $\mathcal{O}(n^{3-\eps})$ under "SETH" \cite[Cor.\ 1.4]{williamsFasterDecisionFirstorder2014}
    \end{tabular} &
    no $\mathcal{O}(n^{4-\eps})$ under "HyperClique" &
    \multicolumn{1}{c}{no $\mathcal{O}(n^{k-\eps})$ (\Cref{proposition: trivial lowerbound})}\\
    \hline
    \multirow{2}{*}{\begin{tabular}{c}
    "PNF" $\fo^{4}$\\(combined)
    \end{tabular}}
    & \begin{tabular}{c}
        $\mathcal{O}(\len{\fml} \cdot n^{4})$ (trivial)\\
        $\mathcal{O}(2^{\len{\fml}} \cdot n)$ (\Cref{proposition: unary})
    \end{tabular}
    & \begin{tabular}{c}
        $\mathcal{O}(\len{\fml} \cdot n^{4})$ (trivial)\\
        $\mathcal{O}(2^{\len{\fml}} \cdot n^{1 + \omega})$ \cite[Cor.\ 1.3]{williamsFasterDecisionFirstorder2014}\\
        \textcolor{red}{$\mathcal{O}(2^{\len{\fml}} \cdot n^{\omega_3})$}
    \end{tabular}
    & \begin{tabular}{c}
        $\mathcal{O}(\len{\fml} \cdot n^{4})$ (trivial)\\
        \textcolor{red}{$\mathcal{O}(2^{\len{\fml}} \cdot n^{\omega_3})$}
    \end{tabular}
    & $\mathcal{O}(\len{\fml} \cdot n^{4})$ (trivial)\\
    \cline{2-5}
    & \multicolumn{4}{c}{no $\poly(\len{\fml}) \cdot n^{4 - \eps}$ under "SETH" (\Cref{theorem: FO3 model checking hardness})}\\
    \hline
    \multirow{2}{*}{\begin{tabular}{c}
    $\fo^{4}$ \\(combined)
    \end{tabular}}
    & \begin{tabular}{c}
        $\mathcal{O}(\len{\fml} \cdot n^{4})$ (trivial)\\
        $\mathcal{O}(2^{\len{\fml}} \cdot n)$ (\Cref{proposition: unary})
    \end{tabular}
    & \begin{tabular}{c}
        $\mathcal{O}(\len{\fml} \cdot n^{4})$ (trivial)\\
        \textcolor{red}{$\mathcal{O}(2^{\len{\fml}} \cdot n^{\omega_3})$}
    \end{tabular}
    & \begin{tabular}{c}
        $\mathcal{O}(\len{\fml} \cdot n^{4})$ (trivial)\\
        \textcolor{red}{$\mathcal{O}(2^{\len{\fml}} \cdot n^{\omega_3})$}
    \end{tabular}
    & $\mathcal{O}(\len{\fml} \cdot n^{4})$ (trivial)\\
    \cline{2-5}
    & \multicolumn{4}{c}{no $\poly(\len{\fml}) \cdot n^{4 - \eps}$ under "SETH" (\Cref{theorem: FO3 model checking hardness})}
  \end{tabular}
}
\caption{Complexity for $\fo^{4}$.}
\label{table: FO4}
\end{table}

% \begin{center}
% \scalebox{.5}{
%   \begin{tabular}{c||c||c|c|c||c}
%     Problem / max. arity $k$ & 1 & 2 & 3 & 4 & 5\\
%     \hline
%     \multirow{2}{*}{\begin{tabular}{c}
%     "PNF" $\fo^{5}$ (data)\\
%     $\fo^{5}$ (data)
%     \end{tabular}} & $\mathcal{O}(n)$ (\Cref{proposition: unary}) & 
%     \begin{tabular}{c}
%         \textcolor{red}{$\mathcal{O}(n^{2 + \omega})$} \cite[Cor.\ 1.3]{williamsFasterDecisionFirstorder2014} (?? for $\fo^{5}$) \\
%         \textcolor{red}{$\mathcal{O}(n^{1 + \omega_3})$} (?? for $\fo^{5}$) \\
%         \textcolor{red}{$\mathcal{O}(n^{\omega_4})$} \\
%     \end{tabular}
%     & \begin{tabular}{c}
%         \textcolor{red}{$\mathcal{O}(n^{1 + \omega_3})$} \cite[Cor.\ 1.3]{williamsFasterDecisionFirstorder2014} (?? for $\fo^{5}$) \\
%         \textcolor{red}{$\mathcal{O}(n^{\omega_4})$} \\
%     \end{tabular} & \textcolor{red}{$\mathcal{O}(n^{\omega_4})$} & $\mathcal{O}(n^{5})$ (trivial) \\  
%     \cline{2-6}
%     & no $\mathcal{O}(n^{1-\eps})$ (\Cref{proposition: trivial lowerbound}) &
%     \multicolumn{2}{c|}{no $\mathcal{O}(n^{4-\eps})$ under "SETH" \cite[Cor.\ 1.4]{williamsFasterDecisionFirstorder2014}} &
%     \multicolumn{2}{c}{no $\mathcal{O}(n^{k-\eps})$ (\Cref{proposition: trivial lowerbound})}\\
%     \hline
%     \multirow{2}{*}{\begin{tabular}{c}
%     "PNF" $\fo^{5}$ (combined)\\
%     $\fo^{5}$ (combined)
%     \end{tabular}}
%     & \begin{tabular}{c}
%         $\mathcal{O}(\len{\fml} \cdot n^{5})$ (trivial)\\
%         $\mathcal{O}(2^{\len{\fml}} \cdot n)$ (\Cref{proposition: unary})
%     \end{tabular}
%     & \begin{tabular}{c}
%         $\mathcal{O}(\len{\fml} \cdot n^{5})$ (trivial)\\
%         \textcolor{red}{$\mathcal{O}(2^{\len{\fml}} \cdot n^{2 + \omega})$} \cite[Cor.\ 1.3]{williamsFasterDecisionFirstorder2014} (?? for $\fo^{5}$)\\
%         \textcolor{red}{$\mathcal{O}(2^{\len{\fml}} \cdot n^{1 + \omega_3})$} (?? for $\fo^{5}$)\\
%         \textcolor{red}{$\mathcal{O}(2^{\len{\fml}} \cdot n^{\omega_4})$}
%     \end{tabular}
%     & \begin{tabular}{c}
%         $\mathcal{O}(\len{\fml} \cdot n^{5})$ (trivial)\\
%         \textcolor{red}{$\mathcal{O}(n^{1 + \omega_3})$} (?? for $\fo^{5}$)\\
%         \textcolor{red}{$\mathcal{O}(2^{\len{\fml}} \cdot n^{\omega_4})$}
%     \end{tabular}
%     & 
%     \begin{tabular}{c}
%         $\mathcal{O}(\len{\fml} \cdot n^{5})$ (trivial)\\
%         \textcolor{red}{$\mathcal{O}(2^{\len{\fml}} \cdot n^{\omega_4})$}
%     \end{tabular}
%     &
%     \begin{tabular}{c}
%         $\mathcal{O}(\len{\fml} \cdot n^{5})$ (trivial)\\
%     \end{tabular}\\
%     \cline{2-6}
%     & \multicolumn{5}{c}{no $\poly(\len{\fml}) \cdot n^{5 - \eps}$, under "SETH" (\Cref{theorem: FO3 model checking hardness})}\\ 
%   \end{tabular}
% }
% \end{center}

\begin{yoshiki}
\begin{itemize}
    \item \Ja{赤の箇所は未証明だが成り立つと思うもの}
    \item \Ja{一般 "$\fo^{k}$" の Williams' algorithm \cite[Theorem 7]{gaoFineGrainedComplexityStrengthenings2019} は怪しい(後述)ので除外}
\end{itemize}
\end{yoshiki}

\subsection{Boolean matrix multiplication and complexity of the model checking}
We recall that
\[(A \cdot B)(x, z) \quad\leftrightarrow\quad \exists y, A(x, y) \land B(y, z).\]
Thus, the "boolean matrix multiplication" can be expressed as a "model checking problem" for dyadic $\fo^{3}$.
Hence, we can observe the following fact.
\begin{proposition}\label{proposition: FO3 model checking and matrix multiplication}
  The "model checking problem" for dyadic $\fo^{3}$ is solvable in $\mathcal{O}(n^{\omega})$ time
  "iff"
  "boolean matrix multiplication" can be computed in $\mathcal{O}(n^{\omega})$ time.
\end{proposition}
\begin{proof}
  $\Longrightarrow$: 
  Matrix multiplication problem can be expressed as a "model checking problem".
  $\Longleftarrow$: By the FO3-to-CoR translation.
\end{proof}

Below is a generalization of the "boolean matrix multiplication" to $k$-tensors.
\begin{definition}[{\cite{lincolnTightHardnessShortest2018}}]\label{definition: k-wise matrix product}
  Given $k$ $k$-tensors of dimensions $n \times \dots \times n$,
  $A_1, \dots, A_k$, 
  the ""$k$-wise matrix product"" is defined as the $k$-tensor $C$ given by:
  \[C[i_1, \dots, i_k] \;\defeq\; \bigvee_{\ell = 1}^{n} \bigwedge_{j = 1}^{k} A_j[i_1, \dots i_{j-1}, i_{j+1}, \dots, i_{k}, \ell].\]
\end{definition}

\begin{proposition}\label{proposition: FOk model checking and matrix multiplication}
  The "model checking problem" for $k$-adic $\fo^{k+1}$ is solvable in $\mathcal{O}(n^{\omega_k})$ time
  "iff"
  "$k$-wise matrix product" can be computed in $\mathcal{O}(n^{\omega_k})$ time.
\end{proposition}
\begin{proof}
  $\Longrightarrow$: 
  "$k$-wise matrix product" can be expressed as a "model checking problem".
  $\Longleftarrow$: By an analog of the FO3-to-CoR translation.(TODO:)
\end{proof}

By a specific restriction, we can give a matrix multiplication characterization for $m$-adic $\fo^{k+1}$,
as follows:
\begin{proposition}\label{proposition: FOmk model checking and matrix multiplication}
  The "model checking problem" for $m$-adic $\fo^{k+1}$ is solvable in $\mathcal{O}(n^{\omega_{k,m}})$ time
  "iff"
  "$k$-wise matrix product", where the input tensors are $\fo^{k+1}$-defined with $m$-ary relations,
   can be computed in $\mathcal{O}(n^{\omega_{k,m}})$ time.
\end{proposition}
\begin{proof}
  $\Longrightarrow$: This restricted "$k$-wise matrix product" can be expressed as a "model checking problem" of $m$-adic $\fo^{k+1}$.
  $\Longleftarrow$: Almost trivial.
\end{proof}

\begin{itemize}
    \item \Ja{Fact: $\omega_{k,k} = \omega_{k}$。とくに $\omega_{2,2} = \omega$。}
    \item \Ja{Fact: $\omega_{k,1} = k$ (成分ごと独立なので)}
    \item \Ja{Fact: $k = \omega_{k, 1} \le \omega_{k,2} \le \dots \le \omega_{k,k} = \omega_{k} \le k+1$}
    \item \Ja{Fact: $\omega_{k} = k+1$, under "HyperClique Hypothesis" \cite{bringmannFineGrainedAnalogueSchaefers2019}, for $k \ge 3$}
\end{itemize}

\subsection{Hyperclique hypothesis and boolean matrix multiplication}

% \begin{hypothesis}["$k$-uniform $\ell$-hyperclique hypothesis" {\cite[Hypothesis 1.4]{lincolnTightHardnessShortest2018}}]\label{hypothesis: hyperclique hypothesis}
% For $\ell > k$ let us consider the following problem:
% Given a "structure" $\struc$ with one (symmetric) $k$-ary relation $E$ (of $k$-tuples of pairwise distinct elements),
% \[\exists v_1, \dots v_\ell \in \univ{\struc},\  \forall S \subseteq \set{v_1, \dots, v_{\ell}} \text{ s.t. $\# S = k$},\  E^{\struc} \cap S^k \neq \emptyset?\]
% That is, this problem asks whether there is a "$k$-uniform $\ell$-hyperclique" in the given "structure"?
% For $\ell > k \ge 3$,
% the ""$k$-uniform $\ell$-hyperclique hypothesis"" is that this problem is not solvable in $\mathcal{O}(n^{4 - \eps})$ time for any $\eps > 0$.\lipicsEnd
% \end{hypothesis}
% This problem can be rephrased as follows.
\begin{hypothesis}["$(k+1)$-hyperclique hypothesis" (rephrased) {\cite[Hypothesis 1.4]{lincolnTightHardnessShortest2018}}]\label{hypothesis: hyperclique hypothesis rephrased}
Let us consider the following "formula"
\[\fml \defeq \exists x_1, \dots x_{k+1},\ \bigwedge_{j = 1}^{k} E(x_1, \dots x_{i-1}, x_{i+1}, \dots, x_{k+1}).\]
Given a "structure" $\struc$ with one symmetric $k$-ary relation $E$ (of $k$-tuples of pairwise distinct elements),
does $\struc \models \fml$ hold?
The ""($k$-uniform) $(k+1)$-hyperclique hypothesis"" is that this problem is not solvable in $\mathcal{O}(n^{4 - \eps})$ time for any $\eps > 0$.\lipicsEnd
\end{hypothesis}
For instance, when $k = 3$, the "formula" in \Cref{hypothesis: hyperclique hypothesis rephrased} is expressed as follows:
\[\fml \defeq \exists x_1, x_2, x_3,\  E(x_1, x_2, x_3) \land \exists x_4, E(x_1, x_2, x_4) \land E(x_2, x_3, x_4) \land E(x_3, x_1, x_4).\]
To calculate this "formula", we can use "$k$-wise matrix product" for the part ``$\exists x_4, E(x_1, x_2, x_4) \land E(x_2, x_3, x_4) \land E(x_3, x_1, x_4)$''.
Hence, we have:
\begin{proposition}
  For $k \ge 3$, under ""$(k+1)$-hyperclique hypothesis"", $\omega_k = k+1$.
  Hence, the "model checking problem" for $k$-adic $\fo^{k+1}$ is not solvable in $\mathcal{O}(n^{k + 1 - \eps})$ time.
\end{proposition}

Below, we introduce a stronger version of the "$(k+1)$-hyperclique hypothesis".
\begin{hypothesis}["binary encoded $(k+1)$-hyperclique hypothesis"]\label{hypothesis: binary encoded hyperclique hypothesis}
For $m \ge 2$,
Let us consider the following "formula":
\[\fml \defeq \exists x_1, \dots x_{k+1},\ \bigwedge_{j = 1}^{k} \mathcal{E}(x_1, \dots x_{i-1}, x_{i+1}, \dots, x_{k+1}),\]
where $\mathcal{E}(y_1, \dots, y_k) \defeq \exists z, \bigwedge_{j = 1}^{k} E(z, y_j)$.
Then, given a "structure" $\struc$ with one binary relation $E$,
does $\struc \models \fml$ hold?
The ""binary encoded $(k+1)$-hyperclique hypothesis"" is that this problem is not solvable in $\mathcal{O}(n^{k - \eps})$ time for any $\eps > 0$.\lipicsEnd
\end{hypothesis}
\begin{yoshiki}
  \Ja{$\mathcal{E}(y_1, \dots, y_k)$ を2項関係の$\fo^{k+1}$論理式でエンコードする場合を考えています。}

  \Ja{愚直に \Cref{hypothesis: hyperclique hypothesis rephrased} に帰着しようとすると頂点数が $n^k$ になるため帰着できない}
\end{yoshiki}

\begin{proposition}\label{proposition: binary encoded hyperclique hypothesis}
    For $k \ge 3$, under the ""binary encoded $(k+1)$-hyperclique hypothesis"", $\omega_{k, 2} = k+1$.
    Hence, the "model checking problem" for dyadic $\fo^{k+1}$ is not solvable in $\mathcal{O}(n^{k + 1 - \eps})$ time.
\end{proposition}

\begin{problem}
  \Cref{hypothesis: binary encoded hyperclique hypothesis} fails?
\end{problem}

\begin{yoshiki}
  \Ja{$\leadsto$ $\omega_{k, 2}$ の問題の形で書けるが $\mathcal{O}(n^{k+1-\eps})$ が open なにか問題はありそうでしょうか?}
\end{yoshiki}

% \begin{yoshiki}
%   \begin{itemize}
%     \item \Ja{"SETH" $\Longrightarrow$ Hyperclique Hypothesis(参照 \href{https://cstheory.stackexchange.com/a/32251}{https://cstheory.stackexchange.com/a/32251})}
%     \item \Ja{"SETH" $\Longrightarrow$ ``Strong'' Hyperclique Hypothesis は未検証 (Open Problem)。}\\
%     \Ja{もしこれが成り立つと \cite[Theorem 7]{gaoFineGrainedComplexityStrengthenings2019} は誤り(under "SETH")。}
%   \end{itemize} 
% \end{yoshiki}

% \subsection{An Efficient "CNF"-"SAT"}
% We say that a "CNF" with $n$ \kl(propositional){variables} is ""$k$-decomposable"" if
% there exists a $k$-partition $V_1, \dots, V_k$ of the \kl(propositional){variables} 
% such that
% \begin{itemize}
%   \item $\#V_i = n/k$,
%   \item for each "clause", there is some $V_i$ such that the "clause" has no \kl(propositional){variables} in $V_i$.
% \end{itemize}
% ""$k$-decomposed-CNF-SAT"" is defined as the following problem:
% given a "$k$-decomposable" "CNF" $\fml$ with its $k$-partition $V_1, \dots V_k$ witness for "$k$-decomposable",
% is the "CNF" $\fml$ "satisfiable"?

% \begin{proposition}\label{proposition: 3-decomposed-CNF-SAT}
%   "$k$-decomposed-CNF-SAT" is in $\mathcal{O}(2^{\omega_{k-1}/k \cdot n} \cdot m)$ time,
%   where $\omega_{k-1}$ is the exponent of $(k-1)$-dimension matrix multiplication.\begin{sideyoshiki}\Ja{メモ\ref{memo: FO4} 参照}\end{sideyoshiki}
% \end{proposition}
% \begin{proof}[Proof Sketch]
%   This reduction is inspired by the reduction from "CNF"-"SAT" to the "$k$-dominating set" \cite{patrascuPossibilityFasterSAT2010}.
%   The "$k$-dominating set" is expressed as the $\fo^{k+1}$ "model checking problem" where 
%   the "sentence" is fixed as follows:
%   \[\exists z_1. \dots \exists z_k. \forall w. \bigvee_{i = 1}^{k} E(z_i, w).\]
%   In particular for "$k$-decomposed"-"CNF",
%   we can reduce into the $\fo^{k}$ "model checking problem"
%   where the "sentence" is fixed as follows:
%   \begin{align*}
%     &\exists z_1. \dots. \exists z_k. \bigwedge_{i = 1}^{k} (\forall z_i. P_{V_i}(z_i) \to \bigvee_{j \in \range{1, k}; j \neq i}E(z_j, z_i)).
%   \end{align*}
%   Here, $P_{V_i}(z_i)$ expresses that the "clause" indicated by $z_i$ has no \kl(propositional){variables} in $V_i$.
% \end{proof}

% \begin{yoshiki}
%   \Ja{"CNF"-"SAT" が $2^{\delta n} \poly(m)$ で解けないという仮説 (\Cref{conjecture: CNF-SAT})}\\
%   \Ja{に対して、"$3$-decomposed" ならば $2^{\delta n} \poly(m)$ で解ける}\\
%   \Ja{一般に $\omega_{k-1} < k$ の、"$k$-decomposed" に対して $2^{\delta n} \poly(m)$ で解ける}

%   \Ja{(メモ \ref{memo: FO4} の hardness($\mathcal{O}(n^{k-\eps})$の非存在)側を考えていた時の失敗作です。 $V_1, \dots, V_k$ を与える必要があるので、そんなに面白みはないのかもしれません)}
% \end{yoshiki}