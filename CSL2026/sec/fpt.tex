
\section{FPT on FO3}

We write $\fo^3_{k, w}$ to denote the set of formulas of $\fo^3$ with $k$-predicates and $w$-quantifier-width.

%\begin{restatable}[Number of predicates is a FPT-parameter]{theorem}{thmFOthreeFPT}
\begin{theorem}[\text{$\fo^3_{k, w}$ model checkig is FPT}]
\label{thm:FO3-FPT}
\mbox{}

Let $\varphi : \fo^3_{k, w}$.

For a given model (graph) $\mathcal{G}$,
we can decide if $\mathcal{G} \models \varphi$ in $O(2^{k + w} \cdot |\varphi| \cdot N^{\omega})$ where
\begin{itemize}
\item $N$ is the number of vertices in $\mathcal{G}$; and
\item $\omega$ is a complexity constant for the boolean matrix multiplication (BMM): i.e., $O(N^\omega)$-time is the time complexity for BMM of two $(N, N)$ matrices.\footnote{\Ja{何らかの論文をciteして、今の記録がどれくらいか書いておくと分かりやすい}}
\end{itemize}
\end{theorem}
 %\end{restatable}

\begin{corollary}
If we fix the number of predicates $k$ and width $w$,
the model checking problem $\mathcal{G} \models \varphi$ in $O(|\varphi| \cdot N^{\omega})$ time-complexity where $N$ is the number of nodes in $\mathcal{G}$.
\end{corollary}

\begin{Jcomment}
    $k$をfixしない、素朴な $\set{ (x, y, z) }$ 構築アルゴリズムだと $O(|\varphi| \cdot N^3)$ になって、
    その設定だと Theorem~\ref{thm:FO3-FPT} は 係数 が大きくなりすぎて遅いということを、
    意味のある形で言及した方が良い。
\end{Jcomment}

\Ja{以降、Theorem~\ref{thm:FO3-FPT} の証明をやっていきます。}

\subsection{Proof of Theorem~\ref{thm:FO3-FPT}}

Here we use $P, Q, R \ldots$ to denote binary predicates and $U, \ldots$ for unary predicates.

For a given graph (model) $\mathcal{G}$, we write $N_{\mathcal{G}}$ or simply $N$ to denote the number of vertices of $\mathcal{G}$.

\paragraph*{Our idea: Quantifier elimination (Simple case)}

Our query evaluation strategy is evaluating the innermost (quantifier-free formula) to the outermost along with quantifier eliminating.

Let us consider the following situation:
$$
Qx. \cdots
( Qy. \cdots
  ( \tikzmarknode{Qz}{Qz}. \cdots
    ( Qx. \cdots
      ( \tikzmarknode{Qy}{Qy}. \cdots
        ( \exists\tikzmarknode{Ex}{x} .
          ( 
            P(\tikzmarknode{Ry}{y}, \tikzmarknode{Rx}{x})
            \land          
            R(\tikzmarknode{Px}{x}, \tikzmarknode{Pz}{z})
          )
        )
      )
    )
  )
)
$$
%
% 矢印の描画
\begin{tikzpicture}[
    remember picture,
    overlay,
    arr/.style={-{[bend]Stealth[length=2mm, width=1.5mm]}, shorten <=0.5pt, shorten >=0.5pt}
]
    % P(x,z) の x から ∃x へ (赤色の矢印)
    \draw[arr, red]
        (Px.north) .. controls +(0, 0.4) and +(0, 0.4) .. (Ex.north);

    % R(y,x) の x から ∃x へ (赤色の矢印)
    \draw[arr, red]
        (Rx.north) .. controls +(0, 0.6) and +(0, 0.6) .. (Ex.north);

    % P(x,z) の z から Qz へ (青色の矢印)
    \draw[arr, blue]
        (Pz.north) .. controls +(0, 0.8) and +(0, 0.8) .. (Qz.north);
        
    % R(y,x) の y から Qy へ (緑色の矢印)
    \draw[arr, green!60!black]
        (Ry.north) .. controls +(0, 0.6) and +(0, 0.6) .. (Qy.north);
\end{tikzpicture}
To eliminate $\exists z. P(y, x) \land R(x, z)$, we introduce a new predicate defined as follows:
$$
S(x, y) := \exists z. P(x, z) \land Q(z, y).
$$

It is worth noting that $S$ is just the composed predicate of $P$ and $Q$.
So, we write $P \odot Q$ instead of $S$.

Using this predicate, we can rewrite the above one to the following one:
$$
Qx \cdots
( Qy \cdots
  ( \tikzmarknode{Qz}{Qz} \cdots
    ( Qx.\
      (
        \tikzmarknode{Qy}{Qy}.\ (P \odot R)(\tikzmarknode{ty}{y}, \tikzmarknode{tz}{z})
      )
    )
  )
)
$$

\begin{tikzpicture}[
    remember picture,
    overlay,
    arr/.style={-{[bend]Stealth[length=2mm, width=1.5mm]}, shorten <=0.5pt, shorten >=0.5pt}
]
    % P(x,z) の z から Qz へ (青色の矢印)
    \draw[arr, blue]
        (tz.north) .. controls +(0, 0.6) and +(0, 0.6) .. (Qz.north);
        
    % R(y,x) の y から Qy へ (緑色の矢印)
    \draw[arr, green!60!black]
        (ty.north) .. controls +(0, 0.4) and +(0, 0.4) .. (Qy.north);
\end{tikzpicture}

Continuing this process, if we eventually enter $\forall x. \exists y. T(x, y)$:
\begin{enumerate}
\item building a unary predicate $T_y(x) := \exists y. T(x, y)$ in $O(N^2)$-time, we rewrite it to $\forall x. T_y(x)$.
\item we then simply evaluate $\forall x. T_y(x)$ as $\bigwedge_{1 \leq i \leq N} T_y(i)$ in $O(N)$-time.
\end{enumerate}

\paragraph*{Time-complexity of basic operations}

Here we enumerate the time-complexity of our basic operations, which will be used later:
\begin{itemize}
\item Building a new predicate $Q(x, y) := P(x, y) \bowtie R(x, y)$ ($\mathord{\bowtie} \in \set{ \land, \lor }$) needs $O(N^2)$-time.
%
\item Transposing a predicate $P^T(x, y) := P(y, x)$ needs $O(N^2)$-time.
%
\item Evaluating unary predicates $\mathcal{Q} x. P(x)$ ($\mathcal{Q} \in \set{\forall, \exists}$) needs $O(N)$-time.
%
\item Building a new predicate $Q(x) := \mathcal{Q} y. P(x, y)$ ($\mathcal{Q} \in \set{\exists, \forall}$) needs $O(N^2)$-time.
\end{itemize}

So, the remaining operator is predicate composing $(P \odot Q)(x, y) := \exists z. P(x, z) \land Q(z, y)$ from predicates $P$ and $Q$.
Although a very simple matrix multiplication requires $O(N^3)$-time,
we can build such $P \odot Q$ using fast matrix multiplication algorithms.
\begin{proposition}[Fact]
From two predicates $P$ and $Q$, we can build the composited predicate $P \odot Q$ in $O(N^\omega)$-time.
\end{proposition}

\subsection{Quantifier elimination: General case}

If we encounter terms of the very restricted form like $\exists z. P(x, z) \land Q(z, y)$,
we can replace it by $(P \odot Q)(x, y)$ with eliminating the quantifier occurrence $\exists z$.

In other words, if we encounter more general form of $\exists z. \cdots$, we need to translate it to some adequate form for quantifier elimination.
%
Let us consider the following term:
$$
\exists z. \biggl(
\bigl(P(x, z) \lor P(y, z) \lor P(x, y) \bigr)
\land
\bigl(Q(x, z) \lor Q(x, y)\bigr)
\land
\bigl(R(y, z) \lor R(x, y)\bigr)
\biggr).
$$

We cannot directly apply our quantifier elimination strategy because $\exists$-quantifiers do not distribute on $\land$: i.e.,
$(\exists z. \varphi_1 \land \varphi_2) \neq (\exists z. \varphi_1) \land (\exists z. \varphi_2)$.

To apply our quantifier elimination procedure, DNF is an adequate form.
However, as well-known, converting-to-DNF translation generates an exponential-large expression in the worst case.
For our example, we fist translate the first $(\cdots) \land (\cdots)$ subterm as follows:
$$
\begin{array}{l}
\bigl(P(x, z) \lor P(y, z) \lor P(x, y) \bigr)
\land
\bigl(Q(x, z) \lor Q(x, y)\bigr) \\
\quad \Rightarrow \\
\phantom{\lor}\,
(P(x, z) \land Q(x,z)) \lor
(P(y, z) \land Q(x,z)) \lor
(P(x, y) \land Q(x, z)) \\
\lor\,
(P(x, z) \land Q(x, y)) \lor
(P(y, z) \land Q(x, y)) \lor
(P(x, y) \land Q(x, y)).
\end{array}
$$
We continue to normalize this term with $R(y, z) \lor R(x, y)$
and it generates DNF with 12 bases of the form $P(\_, \_) \land Q(\_, \_)$.

Since our goal is to design an $O(F(k, w) \cdot |\varphi| \cdot N^{\omega})$-time algorithm, we cannot adopt converting-to-DNF transformations.


\subsection{Tseytin-like (?) transformation for General Cases}

As we have seen above, explicit translations to DNF does not work well since it may explode given expression to an exponential size.

We here develop a translation inspired by Tseytin's transformation.\footnote{\url{https://en.wikipedia.org/wiki/Tseytin_transformation}}

\begin{Jcomment}
    実際はインスパイアされたという程でもないんですが、
    とはいえよく知られた構成な気もするので、何かciteできるものがあればしたいという感じ。
\end{Jcomment}

Let us revisit the above example:
$$
\exists z. \biggl(
\bigl(P(x, z) \lor P(y, z) \lor P(x, y) \bigr)
\land
\bigl(Q(x, z) \lor Q(x, y)\bigr)
\land
\bigl(R(y, z) \lor R(x, y)\bigr)
\biggr).
$$

First, we generate all possible \emph{assignments} to predicates.
For our example, we have the following assignments:
$$
\begin{array}{c|ccccccc}
& P(x, z) & P(y, z) & P(x, y) & Q(x, z) & Q(x, y) & R(y, z) & R(x, y) \\\hline
\alpha_1 & 0 & 0 & 0 & 0 & 0 & 0 & 0 \\
\alpha_2 & 0 & 0 & 0 & 0 & 0 & 0 & 1 \\
\alpha_3 & 0 & 0 & 0 & 0 & 0 & 1 & 0 \\
\vdots \\
\alpha   & 1 & 0 & 1 & 0 & 1 & 0 & 0 \\
\vdots \\
\alpha_{\star}   & 1 & 1 & 1 & 1 & 1 & 1 & 1
\end{array}
$$

Let $\mathcal{A}$ be the set of all assignments.

Let $\sem{\Psi}_{\alpha}$ be a boolean value obtained by evaluationg $\Psi$ under an assignment $\alpha$.

For our example expression $\psi$, $\sem{\psi}_{\alpha} = 0$ using $\alpha$ appearing in the above table.

If we change $\alpha$ as $\alpha(R(y,z)) \gets 1$, then $\sem{\psi}_{\alpha} = 1$.

Introducing assignment leads to the following useful proposition.
\begin{proposition}
$$
\Psi(x, y, z) \iff
\bigvee_{\alpha \in \mathcal{A}} \Bigl( \alpha \land \sem{\Psi}_{\alpha} \Bigr).
$$

Especially, let $\mathcal{V} = \set{ \alpha \in \mathcal{A} : \sem{\Psi}_\alpha }$, then we have:
$$
\Psi(x,y, z) \iff \bigvee_{\alpha \in \mathcal{V}} \alpha(x, y, z). \qquad (\star)
$$
\end{proposition}


By the definition of assignments, $\alpha$ takes the following form:
$$
\alpha(x, y, z) \equiv P_1(x, y) \land P_1(x, z) \land P_2(y, z) \land \cdots \land P_k(x, z).
$$
It means that the right-hand side of $(\star)$ takes DNF. Therefore,
$$
\exists z. \Psi(x, y, z) \iff (\exists z. \alpha_1(x, y, z)) \lor (\exists z. \alpha_2(x, y, z)) \lor \cdots \lor (\exists z. \alpha_{|\mathcal{V}|}(x, y, z)).
$$
It suffices to eliminating an $\exists$-quantifier from $\exists z. \alpha(x, y, z)$ for an assignment $\alpha$.

We first divide $\alpha(x, y, z)$ into two parts $\alpha_z$ and $\alpha_{-z}$ as follows:
$$
\alpha(x, y, z) \equiv \underbrace{(P_1(x, z) \land P_2(z, x) \land P_1(y, z) \land \cdots)}_{\alpha_z : z-\text{involving-terms}} \land \underbrace{(P_1(x, y) \land P_1(y, x) \land P_2 \land \cdots)}_{\alpha_{-z} : z-\text{non-involving-terms}}
$$
We now have the following:
$$
\Bigl(\exists z. \alpha(x, y, z)\Bigr) \iff \Bigl(\alpha_{-z}(x, y) \land (\exists z. \alpha_z(x, y, z))\Bigr).
$$

Finally, we transform $\alpha_z$ in the following steps (order):
\begin{enumerate}
\item If $\alpha_z$ has a term $P(z, x)$, we replace it with $\widetilde{P}(x, z)$ where $\widetilde{P}(a, b) = P(b, a)$.
\item Similarly, we change terms of the form $P(y, z)$ to $\widetilde{P}(z, y)$.
\item At this point, 
$$
\alpha_z(x, y, z) \equiv \underbrace{(P_1(x, z) \land P_3(x, z) \land P_4(x, z) \land \cdots)}_{\Psi_x} \land \underbrace{(P_2(z, y) \land P_3(z, y) \land P_5(z, y) \land \cdots)}_{\Psi_y}.
$$
\item Let $\Psi_x$ (resp. $\Psi_y$) be the set of predicates appearing in the above former (resp. latter) part.
\item Let us introduce two new predicates:
$$
\pi(x, z) := \bigwedge_{P \in \Psi_x} P(x, z), \qquad
\lambda(z, y) := \bigwedge_{Q \in \Psi_y} Q(z, y).
$$
\item We reach our goal:
$$
\alpha_z(x, y, z) = \pi(x, z) \land \lambda(z, y).
$$
\end{enumerate}

Using the final form, we obtain the following:
$$
\begin{array}{lcl}
\biggl(\exists z. \alpha(x, y, z)\biggr) & \iff &
\biggl(\alpha_{-z}(x, y) \land (\exists z. \pi(x, z) \land \lambda(z, y))\biggr) \\
& \iff & \biggl(\alpha_{-z}(x, y) \land (\pi \odot \lambda)(x, y)\biggr) \\
& \iff & (\alpha_{-z} \land (\pi \odot \lambda))(x, y)
\end{array}
$$

\subsection{Considering Complexity}
To eliminate an $\exists$-quantifier from an assignment $\alpha$, we introduce a new single predicate $\alpha_{-z} \land (\pi \odot \lambda)$.

Now we have a question:
To eliminate the $\exists$-quantifier from $\exists z. \Psi(x, y, z)$,
from the following characterization
$$
\biggl(\exists z. \Psi(x,y, z)\biggr)
\iff \biggl(\exists z. \bigvee_{\alpha \in \mathcal{V}} \alpha(x, y, z)\biggr)
\iff \bigvee_{\alpha \in \mathcal{V}} \biggl(\exists z. \alpha(x, y, z)\biggr)
\iff \bigvee_{\alpha \in \mathcal{V}} (\alpha_{-z} \land (\pi \odot \lambda))(x, y)
$$
can we avoid to introduce $O(2^k)$ predicates if we have $k$-predicates??
Please recall that each assignment is a choice from $9k$-literals from $\mathcal{L}$ where
$$
\mathcal{L} = \set{ Q(a, b) : a, b \in \set{x, y, z}, Q \in \set{P_1, P_2, \ldots, P_k} }.
$$

\textbf{Yes. We can avoid introducing a large number of predicates} because
we just need to introduce the following single (but large) predicate:
$$
(\alpha^1_{-z} \land (\pi^1 \odot \lambda^1)) \lor (\alpha^2_{-z} \land (\pi^2 \odot \lambda^2)) \lor \cdots \lor (\alpha^n_{-z} \land (\pi^n \odot \lambda^n))
$$
for $\mathcal{V} = \set{ \alpha^1, \alpha^2, \ldots, \alpha^n }$.

Let us consider what happens when repeatedly eliminating quantifiers from the innermost to the outermost:
$$
\begin{array}{l}
\exists y.
\Bigl(
(\exists z. \cdots) \bowtie (\exists z. \cdots) \bowtie \cdots \bowtie (\exists z. \cdots)
\Bigr) \\
\quad \Rightarrow \\
\exists y. \Bigl(
\begin{array}{c}
\text{Q. how many predicates appear in this scope?} \\
\text{A. $k + w$ where $w$ is the quantifier width of $\psi$}
\end{array}
\Bigr) \\
\quad \Rightarrow \\
\text{Eliminating $(\exists y.)$ introduces a single new predicate.}
\end{array}
$$

Since the required number of quantifier elimination (= the quantifier depth) is bounded by the size of a given expression $|\psi|$, we obtain the following our theorem in the end.

% \thmFOthreeFPT*

% 1. 現在の定理カウンタの値を保存
\setcounter{tempTheoremCounter}{\value{theorem}}
% 2. 目的の定理番号より1つ前の値にカウンタを設定
\setcounter{theorem}{\numexpr\getrefnumber{thm:FO3-FPT}-1\relax}
% 3. 再掲出
\begin{theorem}[Restatement]
Let $\varphi : \fo^3_{k, w}$.

For a given model (graph) $\mathcal{G}$,
we can decide if $\mathcal{G} \models \varphi$ in $O(2^{k+w} \cdot |\varphi| \cdot N^{\omega})$.
\end{theorem}
% 4. カウンタを元の値に戻す
\setcounter{theorem}{\value{tempTheoremCounter}}

\section{FO2 model checking}

\begin{Jcomment}
    $\fo^2$の論理式 $\psi$ と、グラフ $\mathcal{G}$ について
    $\mathcal{G} \models \psi$ のモデル検査問題は
    $O(|\psi| \cdot |\mathcal{G}|)$ で解ける気がします。
    ただし、$|\mathcal{G}|$は頂点数ではなく、述語(つまり辺)の定義の記述長も合わせたものです。
    面白そうなら書いても良いかと思うのですが、$\fo^3$の話だけにしたほうが、話がぼやけないならなくても良いかなという気がします。

    Dense graphについては $(x, y)$ の値の集合を構築する方法で $O(|\psi| \cdot N^2)$ になります。
    この $N$ は $\mathcal{G}$ の頂点数の方です。
    ただ、sparse graphの時には $N^2$ は$|\mathcal{G}|$について線形ではないので、そこを改良したいという感じです。

    アイディアは、$\fo^3$の時と比べると、$\exists y. P(x, y) \land Q(y, x) \simeq (P \odot Q)$ の計算について $(P \odot Q)(x, x)$ の diagonal な形しか要求しないので、行列積を全部やる必要が多分なさそうとかそういう感じです。

    Sparse graphの話なので、もしかすると Impagliazzo たちがやっているかもしれません。
\end{Jcomment}

\begin{Jcomment}
    Yo:
    \cite{gaoCompletenessFirstorderProperties2018} Lemma 9.1 Base Case は $O(|\psi| \cdot |\mathcal{G}|)$ ??
\end{Jcomment}